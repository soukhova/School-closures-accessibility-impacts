% Options for packages loaded elsewhere
\PassOptionsToPackage{unicode}{hyperref}
\PassOptionsToPackage{hyphens}{url}
\PassOptionsToPackage{dvipsnames,svgnames,x11names}{xcolor}
%
\documentclass[
default
]{sn-jnl}

\usepackage{amsmath,amssymb}
\usepackage{iftex}
\ifPDFTeX
  \usepackage[T1]{fontenc}
  \usepackage[utf8]{inputenc}
  \usepackage{textcomp} % provide euro and other symbols
\else % if luatex or xetex
  \usepackage{unicode-math}
  \defaultfontfeatures{Scale=MatchLowercase}
  \defaultfontfeatures[\rmfamily]{Ligatures=TeX,Scale=1}
\fi
\usepackage{lmodern}
\ifPDFTeX\else  
    % xetex/luatex font selection
\fi
% Use upquote if available, for straight quotes in verbatim environments
\IfFileExists{upquote.sty}{\usepackage{upquote}}{}
\IfFileExists{microtype.sty}{% use microtype if available
  \usepackage[]{microtype}
  \UseMicrotypeSet[protrusion]{basicmath} % disable protrusion for tt fonts
}{}
\makeatletter
\@ifundefined{KOMAClassName}{% if non-KOMA class
  \IfFileExists{parskip.sty}{%
    \usepackage{parskip}
  }{% else
    \setlength{\parindent}{0pt}
    \setlength{\parskip}{6pt plus 2pt minus 1pt}}
}{% if KOMA class
  \KOMAoptions{parskip=half}}
\makeatother
\usepackage{xcolor}
\setlength{\emergencystretch}{3em} % prevent overfull lines
\setcounter{secnumdepth}{-\maxdimen} % remove section numbering
% Make \paragraph and \subparagraph free-standing
\makeatletter
\ifx\paragraph\undefined\else
  \let\oldparagraph\paragraph
  \renewcommand{\paragraph}{
    \@ifstar
      \xxxParagraphStar
      \xxxParagraphNoStar
  }
  \newcommand{\xxxParagraphStar}[1]{\oldparagraph*{#1}\mbox{}}
  \newcommand{\xxxParagraphNoStar}[1]{\oldparagraph{#1}\mbox{}}
\fi
\ifx\subparagraph\undefined\else
  \let\oldsubparagraph\subparagraph
  \renewcommand{\subparagraph}{
    \@ifstar
      \xxxSubParagraphStar
      \xxxSubParagraphNoStar
  }
  \newcommand{\xxxSubParagraphStar}[1]{\oldsubparagraph*{#1}\mbox{}}
  \newcommand{\xxxSubParagraphNoStar}[1]{\oldsubparagraph{#1}\mbox{}}
\fi
\makeatother


\providecommand{\tightlist}{%
  \setlength{\itemsep}{0pt}\setlength{\parskip}{0pt}}\usepackage{longtable,booktabs,array}
\usepackage{calc} % for calculating minipage widths
% Correct order of tables after \paragraph or \subparagraph
\usepackage{etoolbox}
\makeatletter
\patchcmd\longtable{\par}{\if@noskipsec\mbox{}\fi\par}{}{}
\makeatother
% Allow footnotes in longtable head/foot
\IfFileExists{footnotehyper.sty}{\usepackage{footnotehyper}}{\usepackage{footnote}}
\makesavenoteenv{longtable}
\usepackage{graphicx}
\makeatletter
\def\maxwidth{\ifdim\Gin@nat@width>\linewidth\linewidth\else\Gin@nat@width\fi}
\def\maxheight{\ifdim\Gin@nat@height>\textheight\textheight\else\Gin@nat@height\fi}
\makeatother
% Scale images if necessary, so that they will not overflow the page
% margins by default, and it is still possible to overwrite the defaults
% using explicit options in \includegraphics[width, height, ...]{}
\setkeys{Gin}{width=\maxwidth,height=\maxheight,keepaspectratio}
% Set default figure placement to htbp
\makeatletter
\def\fps@figure{htbp}
\makeatother
% definitions for citeproc citations
\NewDocumentCommand\citeproctext{}{}
\NewDocumentCommand\citeproc{mm}{%
  \begingroup\def\citeproctext{#2}\cite{#1}\endgroup}
\makeatletter
 % allow citations to break across lines
 \let\@cite@ofmt\@firstofone
 % avoid brackets around text for \cite:
 \def\@biblabel#1{}
 \def\@cite#1#2{{#1\if@tempswa , #2\fi}}
\makeatother
\newlength{\cslhangindent}
\setlength{\cslhangindent}{1.5em}
\newlength{\csllabelwidth}
\setlength{\csllabelwidth}{3em}
\newenvironment{CSLReferences}[2] % #1 hanging-indent, #2 entry-spacing
 {\begin{list}{}{%
  \setlength{\itemindent}{0pt}
  \setlength{\leftmargin}{0pt}
  \setlength{\parsep}{0pt}
  % turn on hanging indent if param 1 is 1
  \ifodd #1
   \setlength{\leftmargin}{\cslhangindent}
   \setlength{\itemindent}{-1\cslhangindent}
  \fi
  % set entry spacing
  \setlength{\itemsep}{#2\baselineskip}}}
 {\end{list}}
\usepackage{calc}
\newcommand{\CSLBlock}[1]{\hfill\break\parbox[t]{\linewidth}{\strut\ignorespaces#1\strut}}
\newcommand{\CSLLeftMargin}[1]{\parbox[t]{\csllabelwidth}{\strut#1\strut}}
\newcommand{\CSLRightInline}[1]{\parbox[t]{\linewidth - \csllabelwidth}{\strut#1\strut}}
\newcommand{\CSLIndent}[1]{\hspace{\cslhangindent}#1}

%%%% Standard Packages

\usepackage{graphicx}%
\usepackage{multirow}%
\usepackage{amsmath,amssymb,amsfonts}%
\usepackage{amsthm}%
\usepackage{mathrsfs}%
\usepackage[title]{appendix}%
\usepackage{xcolor}%
\usepackage{textcomp}%
\usepackage{manyfoot}%
\usepackage{booktabs}%
\usepackage{algorithm}%
\usepackage{algorithmicx}%
\usepackage{algpseudocode}%
\usepackage{listings}%

%%%%

\raggedbottom
\usepackage{fontspec}
\usepackage{multirow}
\usepackage{multicol}
\usepackage{colortbl}
\usepackage{hhline}
\newlength\Oldarrayrulewidth
\newlength\Oldtabcolsep
\usepackage{longtable}
\usepackage{array}
\usepackage{hyperref}
\usepackage{float}
\usepackage{wrapfig}
\makeatletter
\@ifpackageloaded{caption}{}{\usepackage{caption}}
\AtBeginDocument{%
\ifdefined\contentsname
  \renewcommand*\contentsname{Table of contents}
\else
  \newcommand\contentsname{Table of contents}
\fi
\ifdefined\listfigurename
  \renewcommand*\listfigurename{List of Figures}
\else
  \newcommand\listfigurename{List of Figures}
\fi
\ifdefined\listtablename
  \renewcommand*\listtablename{List of Tables}
\else
  \newcommand\listtablename{List of Tables}
\fi
\ifdefined\figurename
  \renewcommand*\figurename{Figure}
\else
  \newcommand\figurename{Figure}
\fi
\ifdefined\tablename
  \renewcommand*\tablename{Table}
\else
  \newcommand\tablename{Table}
\fi
}
\@ifpackageloaded{float}{}{\usepackage{float}}
\floatstyle{ruled}
\@ifundefined{c@chapter}{\newfloat{codelisting}{h}{lop}}{\newfloat{codelisting}{h}{lop}[chapter]}
\floatname{codelisting}{Listing}
\newcommand*\listoflistings{\listof{codelisting}{List of Listings}}
\makeatother
\makeatletter
\makeatother
\makeatletter
\@ifpackageloaded{caption}{}{\usepackage{caption}}
\@ifpackageloaded{subcaption}{}{\usepackage{subcaption}}
\makeatother

\ifLuaTeX
  \usepackage{selnolig}  % disable illegal ligatures
\fi
\usepackage{bookmark}

\IfFileExists{xurl.sty}{\usepackage{xurl}}{} % add URL line breaks if available
\urlstyle{same} % disable monospaced font for URLs
\hypersetup{
  pdftitle={10 years of school closures and consolidations: the impact on school accessibility},
  pdfauthor={Anastasia Soukhov; Christopher D. Higgins; Antonio Páez; Moataz Mohamed},
  pdfkeywords={accessibility; spatial availability; carbon emissions;
journey-to-school; walkability; active transportation; school
consolidation policy; service provision thresholds},
  colorlinks=true,
  linkcolor={blue},
  filecolor={Maroon},
  citecolor={Blue},
  urlcolor={Blue},
  pdfcreator={LaTeX via pandoc}}


\title[10 years of school closures and consolidations: the impact on
school accessibility]{10 years of school closures and consolidations:
the impact on school accessibility}

% author setup
\author*[1]{\fnm{Anastasia} \sur{Soukhov}}\email{soukhoa@mcmaster.ca}\author[2]{\fnm{Christopher D.} \sur{Higgins}}\email{cd.higgins@utoronto.ca}\author[1]{\fnm{Antonio} \sur{Páez}}\email{paezha@mcmaster.ca}\author[3]{\fnm{Moataz} \sur{Mohamed}}\email{mmohame@mcmaster.ca}
% affil setup
\affil[1]{, \orgaddress{\street{School of Earth, Environment and
Society, McMaster University, 1241 Main St.~West, Hamilton, ON, L8S 4K1,
Canada}}}
\affil[2]{, \orgaddress{\street{Department of Human Geography,
University of Toronto Scarborough, 1265 Military Trail, Toronto, ON, M1C
1A4, Canada}}}
\affil[3]{, \orgaddress{\street{Department of Civil Engineering,
McMaster University, 1241 Main St.~West, Hamilton, ON, L8S 4K1,
Canada}}}

% abstract 

\abstract{Reducing motorized travel and increasing active travel are top
priorities for many communities. However, these priorities clash against
many government's reluctance to invest in public services, including
schools. Public school boards are increasingly asked to do more with
less, a directive often accomplished by closing and consolidating
schools. Alas, this policy tends to reduce the proximity of schools for
many students, thus reducing the chances for active travel. In this
paper we study the City of Hamilton, a mid-sized city in Ontario,
Canada, which experienced numerous elementary schools closures between
2011 and 2021. Our analysis evaluates how school closures and
consolidations reshaped accessibility for elementary school students by
motorized and active travel. In this task we use spatial availability, a
singly-constrained multimodal measure of spatial accessibility. This
method's proportional allocation mechanisms allow us to make consistent
comparisons in accessibility between time periods. Analysis is conducted
at the parcel level, considering on-the-ground-capacity (OTGC) of
schools and observed home-to-school trip lengths for motorized and
active modes. Our findings reveal that overall spatial availability in
the city declined, and while the school consolidation policy pursued by
the province may have achieved cost savings in facility operation and
maintenance, it imposes ongoing costs on students, families, and
communities through reduced opportunities for physical activity and
entrenched reliance on motorized travel.}

% keywords
\keywords{accessibility; spatial availability; carbon emissions;
journey-to-school; walkability; active transportation; school
consolidation policy; service provision thresholds}

\begin{document}
\maketitle


\section{Introduction}\label{introduction}

Public services accessible within communities tend to oscillate:
economic and demographic shifts, rural and urban configurations and
political in/action all induce change. These forces spur pressures to
reorganize and consolidate public facilities, potentially improving
access for some but not for all residents in a community (Christiaanse
2020; Rosik, Puławska-Obiedowska, and Goliszek 2021). Public schools are
a compelling case study in this respect. Across the globe, school
closure and consolidation policies have been pursued by governments to
reduce the cost of running public education systems (Rong et al. 2022;
Dai et al. 2019). Research has highlighted that these polices have
knock-on implications, including the weakening of communities' social
infrastructure (Butler, Kane, and Cooligan 2019; Irwin and Seasons 2012;
Autti and Hyry-Beihammer 2014) and negative household-level fiscal
impacts such as a reduction of property values (Merrall, Higgins, and
Páez 2024). Operationally, savings may be experienced by governments,
but they are paid at different levels by students, families,
communities, and ultimately society in general.

A school offers many things to a community, and one relevant dimension
is how it shapes travel. This dimension is often overlooked by
decision-makers tasked with implementing school closure and
consolidation policies (Bierbaum, Karner, and Barajas 2021; J. Lee and
Lubienski 2017). When schools and residential locations are co-located,
the journey-to-school presents an excellent opportunity for physical
activity (Desjardins et al. 2024). However, school closures exchange the
opportunities for active trips by students (Morency, Roorda, and Demers
2009; Morency, Demers, and Lapierre 2007) for the near certainty of
motorized travel (Rong et al. 2022). Indeed, motorized travel to school
is the largest contributor of carbon emissions from personal daily
travel after the journey-to-work (Rong et al. 2022; Pantelaki, Claudia
Caspani, and Maggi 2024). Alas, these impacts are not often considered
in policy when schools close and/or consolidate (Sageman 2022), a
relatively common occurrence globally. For instance, 65\% of
comprehensive schools (e.g.~grades 1 through 9) in rural Finland since
1990 closed (Autti and Hyry-Beihammer 2014), 10\% of elementary schools
(kindergarten through grade 8) in Chicago, USA in 2013 (J. Lee and
Lubienski 2017), and 7\% of primary and lower secondary schools
(kindergarten through grade 9) were closed in Denmark between 2010-2011
(Beuchert et al. 2018).

In this paper, we investigate the case of Hamilton, a mid-sized city in
Canada that has been forced by the province of Ontario to pursue school
closure and consolidation policies. We quantify the transportation
implications of these policies using the capacity of schools in 2011,
2016 and 2021, after several waves of school closures and
consolidations. To this end, we use a novel multimodal extension of a
singly-constrained accessibility measure known as spatial availability
(Soukhov et al. 2023; Soukhov et al. 2024). We use this measure to
estimate active (i.e., walking) and motorized (i.e., car) accessibility
while constraining the supply of the service to the capacity schools. In
this way, this work furnishes evidence to document how government
austerity hurts public goals (such as physical activity) and locks
school districts and families into motorized patterns of travel for the
foreseeable future.

To foreshadow the paper's conclusions, our findings indicate that school
seats have became less available in the city. This change happened
unevenly, with certain communities being disproportionately affected.
The school closure and consolidation policy pursued by the government
may have achieved cost savings in facility operation and maintenance.
However, it has come at a cost to families, students, and communities,
resulting in lost opportunities for healthier life-styles and an
inevitable increase in motorized travel (Hammer and Baluja 2012; Rose
2013; BUCK 2019). To increase the diffusion of this analysis, all work
is transparent and reproducible, following best practices in the spatial
sciences (Brunsdon and Comber 2021; Antonio Páez 2021). For this reason,
all associated code and data are publicly available within the lead
author's
\href{https://github.com/soukhova/School-closures-accessibility-impacts}{GitHub
repository}.

\section{Background}\label{background}

\subsection{Overview of proximity and equity in the school accessibility
literature}\label{overview-of-proximity-and-equity-in-the-school-accessibility-literature}

The examination of accessibility to public services is well-established,
including studies of parks (Reyes, Paez, and Morency 2014; El-Murr et
al. 2021), healthcare (Pereira, Braga, et al. 2021), and public
transportation (Moniruzzaman and Paez 2012; Tiznado-Aitken, Munoz, and
Hurtubia 2021), among many other services (e.g., Bittencourt and
Giannotti 2023; Kelobonye et al. 2020).

Schools are a service that has been studied across several
transportation dimensions. For instance, Marques, Wolf, and Feitosa
(2020) demonstrate that lower-income neighbourhoods in Portugal tend to
have lower accessibility to schools compared to more advantaged
neighbourhoods. Similar conclusions are drawn in West Virginia, US by
Talen (2001), England by Burgess et al. (2011), Baton Rouge, US by
Williams and Wang (2014), São Paulo and Curitiba, Brazil by
Moreno-Monroy, Lovelace, and Ramos (2018), Pizzol, Giannotti, and
Tomasiello (2021), and Bittencourt and Giannotti (2023), and Santiago,
Chile by Tiznado-Aitken, Munoz, and Hurtubia (2021). Research has also
demonstrated that school closures are seldom politically neutral
decisions. For instance, spatial trends were explored in the context of
England by Pinch (1987), areas of declining school-aged populations in
Dreseden, Germany by Müller (2011), rural Finnish communities by Autti
and Hyry-Beihammer (2014), and Chicago, US by J. Lee and Lubienski
(2017).

The methods applied to measure school accessibility have also been
varied. For example, Marques, Wolf, and Feitosa (2020) present
indicators of school seats per student along with the shortest distance
from origin to schools within a school catchment. Similar methods are
also used in Talen (2001) and Burgess et al. (2011). Several authors
propose refinements to these methods; Müller (2011), for instance,
considers travel time by public transit from origins to schools, whereas
Pizzol, Giannotti, and Tomasiello (2021) considers travel times by
different modes and multiple schools from a single origin along with
other indicators related to school quality. By contrast, Williams and
Wang (2014), J. Lee and Lubienski (2017), Moreno-Monroy, Lovelace, and
Ramos (2018), Tiznado-Aitken, Munoz, and Hurtubia (2021), and
Bittencourt and Giannotti (2023) use more complex approaches such as the
two-step floating catchment area (2SFCA). The 2SFCA is popular in
measure in evaluating access to healthcare services (Antonio Páez,
Higgins, and Vivona 2019) and yields a supply-to-demand ratio that
accounts for the population demanding a service and the quantity that is
supplied within a certain distance range (Shen 1998; Luo and Wang 2003).
The consideration of the supply of service is pertinent in the case of
schools and is acknowledged to as a limitation to using simpler
accessibility measures (Pizzol, Giannotti, and Tomasiello 2021).

In addition to competition for school-seats, theoretically, the use of
faster modes (e.g., motorized modes) offers an advantage in terms of a
wider range in school-seat choice. Though some of the works reviewed
consider multiple modes, explicit attention to the modal
\emph{advantage} has not been discussed. To this end, in this work we
use spatial availability, a singly-constrained measure of spatial
accessibility (Soukhov et al. 2023; Soukhov et al. 2024). Spatial
availability allows for the representation of the number of
`school-seats' that are spatially available to students by mode of
transportation (i.e., motorized and active travel), considering the
school-aged student population by zone. Further, spatial availability
can be represented as school-seats per capita, mathematically equivalent
to the supply-to-demand ratio of the 2SFCA (see appendix in Soukhov et
al. (2023)), which increases the interpretability of the analysis, and
the comparability between years.

\subsection{Hamilton's school closures and
consolidation}\label{hamiltons-school-closures-and-consolidation}

Hamilton is a mid-size city (\textasciitilde570,000 pop) in the province
of Ontario, Canada, with rural, suburban and urban characteristics
(Government of Canada 2022a). The city's public English education system
is managed by the Hamilton-Wentworth District School Board (HWDSB) and
the Hamilton-Wentworth Catholic District School Board (HWCDSB). These
two boards, like many in the province have faced fiscal pressures from
the province. For example, in 2013, the HWDSB released its Long Term
Facilities Master Plan (HWDSB 2013), which nominally meant to ensure
``equitable, affordable and sustainable learning facilities''. This plan
indicated that 80\% of its elementary schools (attended by students aged
approximately 5 to 14) would be subject to evaluations, including a
``Pupil Accommodation Review'' (HWDSB 2013). The announcement promptly
led to a public outcry, since the outcome of Pupil Accommodation Reviews
historically translated into school closures and consolidations (Craggs
2013; Seasons 2014b).

School boards in Ontario rely on funding from the province which is
distributed through a formula based on pupil enrollment (Mackenzie 2018;
Irwin and Seasons 2012). However, this funding formula has demonstrably
become insufficient to maintain schools in a state of good repair
(Auditor General of Ontario 2015). Further, existing ``top-up'' programs
to assist in operation and maintenance costs (e.g., School Facilities
Operation and Renewal Grant) were entirely phased out by 2018 (TDSB
2024). Against this backdrop, the Pupil Accommodation Review guidelines
outlined a ``community-focused'' process for assessing the value that a
school provides to students, the community, and the local economy
(Seasons 2014b; Ministry of Education 2006). However in practice,
critical residents felt the outcomes of the Pupil Accommodation Reviews
were predetermined (Thompson, Collins, and Dean 2024): older schools
were slated to be closed, and a wholesome breakdown of costs to repair
and maintain schools was not made available (Kleinhuis 2013). The
process introduced unbalanced incentives for school boards: as a
concerned resident pointed at the time, ``{[}b{]}y closing three
schools, the HWDSB would have a strong case with the Ministry of
Education to receive full funding for a new school'' (Kleinhuis 2013).
The Ministry's framework promotes the closing and consolidation of
schools rather than cost-effective solutions for keeping older schools
open (Kleinhuis 2013).

Hamilton's two school boards (HWDSB and HWCDSB), which together provide
public English education to the majority of school-aged children in
Hamilton (FAO 2023), were particularly impacted by fiscal pressures.
While Accommodation Reviews were a province-wide trend during this
period, Hamilton's school boards underwent an unprecedented number of
reviews compared to boards in other Ontario municipalities. (Seasons
2014a). In the wave of Pupil Accommodation Reviews between 2013 and
2016, 12 of the 147 elementary schools closed in Hamilton. Likely in
part due to the unpopularity of school closures, in 2017, the provincial
government introduced a moratorium, which paused any scheduled Pupil
Accommodation Reviews. This created a backlog of Accommodation Reviews
that HWDSB caught up to in subsequent years, when it closed 12 more
schools between 2017 to 2021 (HWDSB 2023). Though elementary school
locations declined by 10\% between 2011-2021, HWDSB intends to conduct
future Pupil Accommodation Reviews as initially scheduled, waiting until
the end of the moratorium before dates are assigned (HWDSB 2023).

A justification for these Pupil Accommodation Reviews are operational
savings based on ``projected'' reductions in student population and
``under-utilized'' school capacity (Craggs 2012; TDSB 2024; OPSBA 2024).
These policies led to overall reductions in school capacity per student:
school seats in the city declined 5\%--but the impacts were felt
unevenly across the city. It should be noted that although closures
dominated the Pupil Accommodation Review process, some schools were
``consolidated'', that is, a school or two were closed and another
school was expanded or opened to accommodate students from closed
schools. Since school locations decreased proportionally more than
school-seats, elementary schools in Hamilton are on average larger than
they once were.

Existing research indicates that schools have a plethora of benefits not
captured by their maintenance and operational costs (Butler, Kane, and
Cooligan 2019; Irwin and Seasons 2012; Autti and Hyry-Beihammer 2014),
and in this paper we focus on the transportation implications. For
students who live in proximity to schools, closures or consolidations
directly eliminate the potential for active trips, and consequently
physical activity. Closing a school permanently reduces the potential
for active trips and further entrenches reliance on motorized travel.
Additionally, from the perspective of modal equity, school-aged
populations with access to motorized transport gain a speed-time
advantage, providing them with greater flexibility and choice in
accessing school-seats.

The spatial and population-level distribution of these impacts are of
question for this paper. Specifically, the paper's aim is to investigate
how these policies impacted the motorized and non-motorized (active)
school-seat accessibility landscape. We hypothesize that school closure
and consolidation policies reduced the availability of school-seats for
many students. By extension, this loss in accessibility and the
displacement of active trips likely resulted in decreased potential
active travel minutes and increased motorized travel minutes, with these
impacts distributed inequitably across space and populations. To test
these hypotheses, the paper examines the following:

\begin{itemize}
\tightlist
\item
  Did spatial availability change for home-to-school trips following
  school closures and consolidations?
\item
  What is the extent of the loss in walking minutes for these trips?
\item
  How has motorized travel increased as a result?
\item
  Which communities, particularly households with a high prevalence of
  low-income and hence more likely experiencing transport poverty,
  experienced the greatest losses due to this policy?
\end{itemize}

\section{Data}\label{data}

We focus on the evaluation of the impacts of school closures and
consolidations between 2011, 2016 and 2021. The outcome of interest is
the multimodal school-seat spatial availability in Hamilton. To research
this, the following data were prepared for each year: 1) school
configurations, 2) student population, prevalence of low-income
households, and residential parcel locations, and 3) estimated trip
length by mode. These data are described next.

\subsection{Schools}\label{schools}

For this study, we use the locations of elementary schools and their
capacities at each of three years: 2011, 2016, and 2021.

Concerning school locations, the City's school boards provided the
authors with locations and associated school catchments for the
2010-2011 and 2015-2016 academic years. School locations in 2021 were
retrieved from the City's open data portal (Hamilton 2024). In Hamilton,
the majority of student-aged population attend school in one of the two
English public school boards (FAO 2023). School catchments for the
public and public-catholic catchments indicate the default school for
households. However, families can decide to attend schools out of
catchment, and in fact, 21\%-23\% of motorized school trips for
populations aged 5 through 14 are out of catchment according to the
regional travel survey (the Transportation Tomorrow Survey, or TTS) in
2011 and 2016, the most recent TTS available at the time of writing
(Data Management Group 2018).

In addition to school locations, all schools were also tagged with a
school status regarding their operational status (active or closed)
and/or changes in location or school-seat capacity. A summary of the
status of schools through 2011 to 2021 is shown in
Figure~\ref{fig-Fig1}. Figure~\ref{fig-Fig1} also displays a count of
the number of schools with the status between years, the number of
schools overall present in each year, and the symbolic representation of
a school status that is used in Figures throughout the paper. Overall, a
school can have one of seven school statuses in any given year. Either:
1) the school did not change throughout 2011 to 2021, 2) the school
expanded (i.e., increased school-seat capacity) or was 3) relocated and
expanded (N.B., schools expanded only between 2011 through 2016 as they
were typically done \emph{before} school closures and no new Pupil
Accommodation Review was commenced after 2017), 4) the school closed
sometime after 2011 and before 2016 or 5) after 2016 and before 2021,
and 6) the school opened sometime after 2011 and before 2016 or 7) after
2016 and before 2021.

\begin{figure}

\centering{

\includegraphics[width=1.2\textwidth,height=\textheight]{../Figures/Fig1.png}

}

\caption{\label{fig-Fig1}Overview of the number of schools in Hamilton
in 2011, 2016 and 2021 along with the type of school status change..}

\end{figure}%

The capacity of each school is also required for each year. This
capacity is the number of `seats' available in the school. Capacity is
calculated by the provincial Ministry of Education and referred to as
the ``on-the-ground capacity'' (OTGC) of a school, albeit it is
important to note that schools rarely operate at their nominal capacity,
and can be either over-enrolled or under-enrolled.

Current and historic OTGCs for schools are not publicly available
(Ontario 2017), so we used a set of known OTGC values gathered from
Pupil Accommodation Review documents and multivariate regression to
estimate and validate these values. Two regression models were created:
one for public schools and another for public catholic schools. The
independent variables in these models were 1) a dummy variable
indicating the level of the school (\emph{MidElem}, \emph{JrElem},
\emph{Elem}); 2) the school's building footprint (\emph{F}) in \(m^2\);
and 3) the school's Euclidean distance from the centroid of Hamilton's
central business district area (\emph{DistCBD}) in metres.

Formally, elementary schools are defined as schools that provide
instruction to any combination of grades between kindergarten to grade 8
(i.e.~typically children aged 5 to 14). As such, elementary includes
middle schools that only instruct grades 6 to 8 (\emph{MidElem}),
primary schools that only instruct kindergarten to grade 5 or 6
(\emph{JrElem}), and all grade elementary schools (\emph{Elem}) which
instruct all grades from kindergarten to grade 8. \emph{MidElem} and
\emph{JrElem} school grade instruction type is only present in the
public school board.

\begin{itemize}
\item
  The building footprint \emph{F} was retrieved from an archived spatial
  data set (Spatial 2015) and footprints from newer schools in 2016 and
  2021 from OSM (OpenStreetMap 2021).
\item
  \emph{DistCBD} was calculated `as the crow flies' from each school to
  the centroid of the Hamilton CBD (43.256684\(\circ\)N,
  79.869039\(\circ\)W). Notably, this variable can also be seen as a
  proxy for school construction age as older buildings in Hamilton are
  generally located closer to the CBD, the ``old'' Hamilton community
  (Merrall 2021).
\end{itemize}

Equations (\ref{eq:OTGC-public}) and (\ref{eq:OTGC-catholic}) are the
OTGC regression model for public schools and public catholic schools
respectively. As seen in Table \ref{TabA1-OTGC}, the coefficient of
determination for these models are quite high, at \(R^2 = 0.999\) and
\(R^2 = 0.998\) for the public and public catholic school boards
respectively, and the residual standard deviation is \(\sigma = 0.212\)
and \(sigma = 0.331\).

\begin{equation}
\label{eq:OTGC-public}
OTGC_{Public} = F^{0.346}-e^{0.00003*DistCBD}+e^{3.123*JrElem}+e^{3.752*Elem}+ e^{3.068*MidElem} 
\end{equation}

\begin{equation}
\label{eq:OTGC-catholic}
OTGC_{Public Catholic} =F^{0.471}-e^{0.00003*DistCBD}+e^{2.333*Elem}
\end{equation}

\input{Tab1.tex}

Taken together, Figure~\ref{fig-Fig2} displays the city's elementary
schools locations, status, and their observed or estimated OTGCs for
2011, 2016 and 2021. For additional context on the degree of residential
urbanization in Hamilton, the percentage of residential parcels
considered `urban' compared to `suburban' or `rural' is reflected at an
aggregated spatial unit across all six plots in Figure~\ref{fig-Fig2}.
This land-use classification is available within the 2021 residential
parcel file described in the following sub-section. In
Figure~\ref{fig-Fig2}, it can be seen that the majority of schools that
changed status are in the HWDSB, with OTGC capacity decreasing in 2016
and 2021 compared to the previous year. Most school closures took place
in the central and eastern parts of Hamilton's urban area (Hamilton
Central), as well as in rural areas of western Hamilton (Flamborough).
When schools did open or expand, it was primarily in the more recently
urbanized southeastern area of Hamilton (Glanbrook), with a few also in
Hamilton Central to offset some of the lost capacity.

\begin{figure}

\centering{

\includegraphics[width=1.2\textwidth,height=\textheight]{Manuscript_files/figure-pdf/fig-Fig2-1.pdf}

}

\caption{\label{fig-Fig2}The on-the-ground capacity (OTGC) of schools
and school status for year 2011, 2016 and 2021 for both the HWDSB and
HWCDSB. Schools are presented overtop a layer visualising the degree of
residential urbanisation in 2021.}

\end{figure}%

\subsection{Students, residential locations and low-income
prevelance}\label{students-residential-locations-and-low-income-prevelance}

Secondly, a detailed account of the average student population,
low-income prevalence of households, and where they may live is required
for each studied year.

Concerning the student population and poverty, information from the
2011, 2016 and 2021 Canadian Census (Canada 2011, 2016, 2021) was
sourced using the \{cancensus\} R package (Bergmann, Shkolnik, and
Jacobs 2021). The census releases population data by age group category,
so the population aged between 5-9 years and 10-14 years were retained.
A common poverty measure across all three census years is the low-income
after-tax measure (LIM-AT), which reflects the proportion of private
households that are below the median after-tax income in the region
(Government of Canada 2017). The LIM-AT prevalence for households with
children under 18 was retrained for this paper as there is no LIM-AT
prevalence tabulated for households with exclusively elementary aged
children. Population and LIM-AT variables were taken at the
Dissemination Area (DA) level, the finest geographic unit publicly
available. DAs are designed by Statistics Canada with the aim of
population uniformity hence DAs greatly vary in area but represent
between approximately 400 and 700 (1st and 3rd quartile) in total
population. The population aged 5 to 14 and the proportion of LIM-AT are
visualised in the first and last rows in Figure~\ref{fig-Fig3}. LIM-AT
prevalence is notably more concentrated within the centre of Hamilton
(Hamilton Central) through 2011 to 2021, overlapping the most urbanised
land-use and the largest amount of schools closed
(Figure~\ref{fig-Fig2}). Also of note, LIM-AT prevalence drastically
decreased in 2021 as a result of Pandemic benefits that reduced income
inequality (Government of Canada 2022b).

Student residential locations were approximated at the level of
residential parcels. Centroids for parcels in 2011, 2016 and 2021 were
retrieved from Teranet (2009), representing \(134,340\), \(139,467\) and
\(143,890\) unique locations, respectively. Each residential unit in a
parcel was populated with the average number of children aged 5-14 in
the corresponding dissemination area (DA). Due to the proprietary nature
of the parcel data, the second row in Figure~\ref{fig-Fig3} shows an
aggregation of the information: the average rate of 5-14 year old
population per residential unit (parcel) at the DA level. Notably,
though the rate of 5-14 year old population per parcel is higher in more
peripheral (and rural) communities, there are many DAs within Hamilton
Central that have high rates and populations that are similar to those
in these more peripheral communities.

\begin{figure}[H]

\centering{

\includegraphics[width=1.2\textwidth,height=\textheight]{Manuscript_files/figure-pdf/fig-Fig3-1.pdf}

}

\caption{\label{fig-Fig3}The magnitude (top row) and rate per
residential unit (middle row) of elementary aged student population per
DA in 2011, 2016 and 2021. Bottom row: the proportion of low-income
households (prevelance of low-income after-tax, LIM-AT, measured by the
2011, 2016 and 2021 Canadian Census) with dependents under 18 years old
shown. All scales are represented in quartiles.}

\end{figure}%

\subsection{Trip length by mode}\label{trip-length-by-mode}

To reflect multimodal travel behaviour, two types of data were estimated
and compiled. The process to do so is explained next.

First, we retrieves information about the mode used and
origin-destination locations of home-to-school trips from the 2011 and
2016 Transportation Tomorrow Survey (TTS) (Data Management Group 2018);
the 2021 TTS survey is unavailable at the time of writing so 2016 flows
are assumed for the 2021 year. The TTS is a travel survey conducted in
the Greater Golden Horseshoe Area in Ontario typically every five years.
With a target 5\% sampling rate, the survey is expanded to be
representative at the traffic analysis zone (TAZ) level of geography.
TAZ are spatial units created for the purpose of the TTS and pulled from
the R data package \{TTS2016R\} (Soukhov and Páez 2023); a few DAs
typically nest within each TAZ. The trip-level travel data extracted for
this paper represent 13,715 and 12,878 motorized trips (mode used
includes private car passenger, school bus, taxi, and transit) and 7,432
and 7,085 non-motorized trips (mode used includes walk and cycling) for
5 to 14 year olds from home-to-school in 2011 and 2016 respectively. The
intensity of modal home-to-school flows are visualised in
Figure~\ref{fig-Fig4} along with the boundaries of the TAZ and DAs. Also
in Figure~\ref{fig-Fig4}, not all TAZs capture an elementary school trip
by both modes. For this reason, the proportion of motorized modal share
is aggregated at the community level. Hamilton (Ancaster - 2011: 78\%and
2016:94\%, Dundas - 2011: 71\% and 2016: 67\%, Flamborough - 2011: 90\%
and 2016: 91\%, Glanbrook - 2011: 91\% and 2016: 78\%, Hamilton Central
- 2011: 55\% and 2016:50\%, and Stoney Creek - 2011: 71\% and 2016:
79\%).

\begin{figure}[H]

\centering{

\includegraphics[width=1.2\textwidth,height=\textheight]{Manuscript_files/figure-pdf/fig-Fig4-1.pdf}

}

\caption{\label{fig-Fig4}The origin-destination flows from home-based
motorized and non-motorized school trips for students between 5-14 years
old as retrieved from TTS 2011 and 2016. Flows are mapped atop the
proportion of TAZ non/motorized modal share from the TTS 2011 and 2016,
DA boundaries, and Hamilton community boundaries.}

\end{figure}%

Next, travel times matrices for motorized and active travel were
calculated using \{r5r\} package (Pereira, Saraiva, et al. 2021). Travel
time matrices were estimated for all TAZ centroids to TAZ centroids
(matrices of size for 2011 and 2016: 234 x 234) as they are not reported
in the TTS. Additionally matrices were estimated for all residential
parcels to schools (matrices of size 134,340 x 147 for 2011, 139,467 x
137 for 2016, and 143,890 x 132 for 2021) for additional granularity.
\{r5r\} is a R-based interface to the R5 routing engine (Conveyal
{[}2015{]} 2022). For simplicity, we assume motorized travel is by car
and non-motorized travel is by walking, as these are the most common
modes in their respective categories. For travel time calculations, we
set a maximum threshold of 60 minutes and use the free-flow
OpenStreetMap road network of Hamilton (Geofabrik 2022). While all car
trips are retained, walking trips over 27 minutes were filtered out.
This value approximately corresponds to a distance of 1.6 km at a
walking speed of 3.6 km/h, which is the threshold for qualifying for
motorized transport provided by the school board (HWDSB 2019).

In terms of the travel impedance function: we assume that children can
go to any elementary school, however, there is a preference for
facilities that are more proximate. Based on this assumption, we matched
the associated TAZ-to-TAZ travel times to all observed student-to-school
travel flows from the TTS; these are visualised as grey bar columns in
Figure~\ref{fig-Fig5}. Using these observed values, the
\emph{theoretical} trip length distribution (TLD) functions are
calibrated. TLDs can be interpreted as travel impedance functions as
they represent the propensity of realized travel, by trip length
(Horbachov and Svichynskyi 2018; Batista, Leclercq, and Geroliminis
2019). The TLDs are calibrated using the maximum likelihood and moment
matching techniques and the Nelder-Mead and Brent methods for direct
optimization available within the \{fitdistrplus\} R package
(Delignette-Muller and Dutang 2015). The theoretical TLD for each mode
and available study year is visualised in blue in Figure~\ref{fig-Fig5}.
Based on goodness-of-fit criteria and diagnostics, the gamma and
exponential distributions were selected for the motorized and
non-motorized modal distributions respectively. The gamma distribution
is defined by the shape (\(\alpha\)) parameter of 1.939 (2011) and 2.046
(2016) and the rate (\(\beta\)) of 0.233 (2011) and 0.236 (2016). The
exponential distribution is defined by the rate (\(\beta\)) parameter of
0.092 (2011) and 0.1 (2016). For reference, the gamma distribution and
the exponential distribution function are displayed in Equations
(\ref{eq:exp-dist}) and (\ref{eq:gamma-dist}) where \(x\) is \(c_{ij}\):

\begin{equation}
\label{eq:exp-dist}
f(x) = \beta e ^{-\beta x}
\end{equation}

\begin{equation}
\label{eq:gamma-dist}
\begin{aligned}
f(x, \alpha, \beta) = \frac {x^{\alpha-1}e^{-\frac{x}{\beta}}}{ \beta^{\alpha}\Gamma(\alpha)} \quad \text{for } 0 \leq x \leq \infty \\
\Gamma(\alpha) =  \int_{0}^{\infty} x^{\alpha-1}e^{-x} \,dx
\end{aligned}
\end{equation}

\begin{figure}[H]

\centering{

\includegraphics[width=1\textwidth,height=\textheight]{Manuscript_files/figure-pdf/fig-Fig5-1.pdf}

}

\caption{\label{fig-Fig5}Observed (grey bars) and theoretical (blue
curves) motorized and non-motorized impedance functions. Motorized
theoretical impedance function is based on the gamma distribution
function and non-motorized on the expoential distribution function}

\end{figure}%

Lastly, to achieve greater granularity, the TLDs calibrated using
TAZ-to-TAZ flows are used to estimate travel impedance values for each
parcel-to-school flow based on calculated parcel-to-school travel times.
These values represent the likelihood that students in a parcel will
travel to a school, informed by observed home-to-school flows from the
TTS.

\section{Methods: Multimodal spatial
availability}\label{methods-multimodal-spatial-availability}

Accessibility is defined as the ``potential for spatial interaction''.
It is classically presented as the gravity-based measure defined in
Hansen (1959) and takes the following general multimodal formulation:

\begin{equation}
\label{eq:multimodal-conventional-accessibility}
S_i^m = \sum_{j=1}^JO_j \cdot f^m(c_{ij}^m)
\end{equation}

\noindent where:

\begin{itemize}
\tightlist
\item
  \(m\) is a set of modes.
\item
  \(c_{ij}^m\) is a measure of the cost of moving between \(i\) and
  \(j\) for each \(m\).
\item
  \(f^m(\cdot)\) is an impedance function of \(c_{ij}^m\) for each
  \(m\); it can take the form of any monotonically decreasing function
  chosen based on positive or normative criteria (A. Páez, Scott, and
  Morency 2012).
\item
  \(i\) is a set of origin locations (\(i = 1,\cdots,N\)).
\item
  \(j\) is a set of destination locations (\(j = 1,\cdots,J\)).
\item
  \(O_j\) is the number of opportunities at location \(j\).
\item
  \(S\) is Hansen-type accessibility as weighted sum of opportunities.
\end{itemize}

As indicators of urban structure, Hansen-type accessibility measures
like Equation (\ref{eq:multimodal-conventional-accessibility}) are
informative in reflecting the magnitude of access but meaning of the
value itself is elusive. The significance of 10,000 accessible
school-seats is hard to pin down: how many opportunities must any single
student have access to? Furthermore, this opaque interpretation
especially is complicated when comparing accessibility of school seats
between years, such as different years as in this work. The
interpretability of Hansen-type accessibility has been discussed in
numerous studies, including recently by Hu and Downs (2019), Kelobonye
et al. (2020), and in greater depth by Merlin and Hu (2017) along with
Soukhov et al. (2023) and Soukhov et al. (2024). The interpretation of
accessibility depends on how many people demand the opportunity,
especially for exclusive opportunity-types like schools-seats (i.e., one
school-seat is for one student).

In this paper, our work benefits from new developments in accessibility
research, particularly the multimodal spatial availability measure
(Soukhov et al. 2023; Soukhov et al. 2024). Spatial availability is a
singly-constrained accessibility measure that accounts for competition
by students using different modes for exclusive opportunities, such as
school seats. The measure's single constraint ensures that the marginals
at the destination are met and thus the number of estimated school seats
(opportunities) are preserved and allocated proportionally to the
mode-using student population. This proportional allocation of
opportunities yields an interpretable and meaningful measure of
opportunity access, particularly when comparing across modes, at the
spatial resolution of a residential parcel, and multiple time periods.
See Soukhov et al. (2024) for further discussion, multimodal spatial
availability \(V_{i}^m\) is defined as given by Equation
(\ref{eq:spatial-availability}):

\begin{equation}
\label{eq:spatial-availability}
V_{i}^{m} = \sum_{j=1}^J O_j\ F^{tm}_{ij}
\end{equation}

\noindent where:

\begin{itemize}
\tightlist
\item
  \(F^{tm}_{ij}\) is a balancing factor that depends on the population
  and cost of movement in the system as part of the gravity modelling
  framework and is captured in Equation (\ref{eq:balancing-factors}) for
  mode \(m\).
\item
  \(O_j\) is the number of opportunities at \(j\).
\item
  \(V_i^m\) is the number of spatially available opportunities from the
  perspective of \(i\) for mode \(m\).
\end{itemize}

\(F^{tm}_{ij}\) can be understood as the joint probability of allocating
opportunities, where \(F^{pm}_{i}\) is the population-based balancing
factor that grants a larger share of opportunities to larger \(m\)
population spatial units and \(F^{cm}_{ij}\) is the impedance-based
balancing factor that grants a larger share of the opportunities to less
\(m\)-travel costly centers. Together \(F^{tm}_{ij}\) ensures
proportional allocation such that opportunities \(O\) (like
school-seats) are preserved for the whole region (i.e.,
\(O = \sum_{j} O_j = \sum_{i} V_i = \sum_{m}\sum_{i} V_i^m\) ) and is
reflected in Equation (\ref{eq:balancing-factors}):

\begin{equation}
\label{eq:balancing-factors}
F^{tm}_{ij} = \frac{F^{pm}_{i} \cdot F^{cm}_{ij}}{\sum_{i} F^{pm}_{i} \cdot F^{cm}_{ij}}
\end{equation}

\noindent where:

\begin{itemize}
\tightlist
\item
  The factor for allocation by population for each \(m\) at each \(i\)
  is \(F^{pm}_{i} = \frac{P_{i}^m}{\sum_{m}\sum_{i} P_{i}^m}\)
\item
  The factor for allocation by travel cost for each \(m\) at each \(i\)
  and \(j\) is
  \(F^{cm}_{ij} = \frac{f^m(c^m_{ij})}{\sum_{m}\sum_{i} f^m(c^m_{ij})}\)
\end{itemize}

It should be noted that, when summed over all spatial units in the
region, the population-based allocation factors \(F^{pm}_{i}\) always
equal 1 (\(\sum_{m}\sum_{i} F^{pm}_{i}= 1\)), likewise for
impedance-based allocation factors \(F^{cm}_{i}\)
(\(\sum_{m}\sum_{i} F^{cm}_{ij} = 1\)).

Hansen-type accessibility is not designed to preserve the number of
opportunities in the region, it simply counts the intensity of
opportunities that those in a zone can potentially interact with
(weighted by the friction of distance). Also, as discussed in Soukhov et
al. (2023), popular competitive accessibility measures such as the
two-step floating catchment area (2SFCA) (Joseph and Bantock 1982;
Weibull 1976; Shen 1998; Luo and Wang 2003) are internally inconsistent,
and the only way it preserves the number of opportunities is if the
effect of the impedance function is ignored when expanding the values of
opportunities per capita to obtain the total number of opportunities. On
the other hand, the proportional allocation procedure associated with
calculating multimodal spatial availability \(V_i^m\) consistently
returns a number of opportunities available to populations by mode that
matches the total number of opportunities in the region when summed. By
doing this consistently, it is possible to define a measure of
multimodal spatial availability per capita as presented in Equation
(\ref{eq:SA-per-capita}) for use as a benchmark to compare against the
regional opportunities per capita
(\(\frac{\sum_{j} O_j}{\sum_{i} P_i}\)).

\begin{equation}
\label{eq:SA-per-capita}
v_i^m = \frac{V_i^m}{P_i^m}
\end{equation}

To summarise the methodology and the data as described in Section 3,
multimodal spatial availability \(V_i^m\) and spatial availability per
student \(v_i^m\) is calculated for each parcel for each studied year as
follows:

\begin{itemize}
\tightlist
\item
  First, all residential parcels are associated to their respective
  census DA, TTS TAZ, and community boundary based on spatial location.
  Each parcel is assigned a number of `potential' motorized and
  non-motorized student aged population (from DA) and modal share (from
  TTS) to calculate a motorized and non-motorized population balancing
  factor \(F_{i}^{pm}\). For each parcel, the elementary-aged student
  population per parcel (Figure~\ref{fig-Fig3} second row) and the
  motorized and non-motorized share as informed by the TTS aggregated by
  community (Figure~\ref{fig-Fig4}) is retained.
\item
  Second, the school capacity \(O_j\) is estimated and associated with
  each school (Figure~\ref{fig-Fig2}). All residential locations are
  assumed to be able to access all schools by non-motorized and/or
  motorized mode. All origins (residential locations) can reach all
  schools by motorized mode, but few residential locations can reach
  schools within a 27 minute non-motorized trip.
\item
  Third, the motorized and non-motorized impedance-balancing factors
  \(F_{ij}^{cm}\) are calculated using a mode-specific impedance
  function (Figure~\ref{fig-Fig5}) based on the respective estimated
  travel times for each origin (parcel) to school combination. Using
  network estimated travel time sensitive to observed parcel origin to
  school destination flows conceptually addresses the aggregation error
  that could result from using less granular zonal units to represent
  origin/destinations (Kane and Kim 2020; Kwan 1998; Hewko,
  Smoyer-Tomic, and Hodgson 2002).
\item
  Finally, outputs from all three stages are joined together.
  \(F_{i}^{pm}\) joined with \(F_{ij}^{cm}\) yields the total balancing
  factor \(F^{tm}_{ij}\) which serves to proportionally allocate the
  school capacity \(O_j\) to each parcel. The value is multimodal
  spatial availability \(V_i^m\) which can be interpreted as the number
  of school-seats that are available to the mode-using population at
  that parcel. \(V_i^m\) is then summed and represented at the DA and
  community level along with the calculated \(v_i^m\) for
  interpretation.
\end{itemize}

\section{Results}\label{results}

The probability density plots of the spatial availability values \(V_i\)
and the per capita spatial availability \(v_i\) values for both modes
and all three years are displayed in Figure~\ref{fig-Fig6} as a
city-wide overview. The left column displays parcel-level values, and
the right reflects parcel-level values aggregated at the 2021 DA grid.
Both the parcel and DA level values reflect the spatial unit at which
they were calculated or aggregated. However, the values of the \(V_i\)
plots are not directly comparable since the x- and y- axes are set at
different scale limits to highlight between year differences. However,
overall trends in their distribution can be compared: in all four
\(V_i\) plots, the 2011 data is more right-skewed than more current
years, indicating that spatial availability values used to be higher
(e.g., more school-seat access). School closures and consolidations had
an impact despite the background dynamics of assumed student population,
mode use, travel time and parcel location. The DA level \(V_i\) plots
show the distribution of each DA value, which is the summation of all
parcel \(V_i\) values within a DA. In 2011, DAs can be observed to also
have more right-skewed spatial availability values than previous years.
In contrast, the 2016 and 2021 distributions are more intense and
heavily skewed toward lower values.

Although \(V_i\) values cannot be directly compared, zonal per capita
\(v_i\) values can be as the school-age population per spatial unit
(e.g., parcel or DA) is used to normalize the respective \(V_i\) value.
\(v_i\) values are presented in the bottom four plots of
Figure~\ref{fig-Fig6}. It can be observed that the same 2011
right-skewed trend persists across all modes. Motorized mode-using
students have access to more spatially available school seats than
non-motorized mode using students (in 2011, approximately 1.3
school-seats per student as compared to 0.7 and in 2021 approximately
1.2 compared to 0.5). Due to school-seats per student at the DA level
being an interpretable aggregation, values at this level will be the
discussion focus of the remainder of this paper.

\begin{figure}[H]

\centering{

\includegraphics[width=1.2\textwidth,height=\textheight]{Manuscript_files/figure-pdf/fig-Fig6-1.pdf}

}

\caption{\label{fig-Fig6}Spatial availability (Vi) and per capita
spatial availability (vi) probability density distributions for the city
of Hamilton in 2011, 2016 and 2021. Aggregated at the parcel level (left
column) and DA level (right column).}

\end{figure}%

Representing the right column of Figure~\ref{fig-Fig6} spatially,
Figure~\ref{fig-Fig7} displays the spatial availability for each DA for
a given mode-using population \(m\) (e.g., the sum of school seat
availability per parcel in each DA for motorized or non-motorized
students). Figure~\ref{fig-Fig7} visualizes the spatial distribution of
motorized and non-motorized school seat availability for 2011, 2016, and
2021. As a reminder, the sum of \(V_i^m\) values for both motorized and
non-motorized populations in a given year equals the city-wide OTGC
(i.e., school seats across both the public english HWDSB and public
english catholic HWCDSB). This is due to \(V_i^m\)'s proportional
allocation mechanism, singly-constraining opportunities through the
proportion allocation balancing factors. In this way, \(V_i^m\) values
can be interpreted as a proportion of the total OTGC for that year, as
the sum of \(V_i^m\) for both modes in a year equals the total OTGC for
that year.

Referring to the first two rows of \(V_i\) plots in
Figure~\ref{fig-Fig7} (purples), it is notable that the magnitudes of
\(V_i\) values for the motorized and non-motorized populations are
tremendously different. Again, each \(V_i\) value reflects the number of
school seats spatially available to the mode-using population at that
zonal level, so due to the lower non-motorized modal share (refer to
Figure~\ref{fig-Fig4}), non-motorized values will be lower. However,
both mode-using populations have higher spatial availability values
(darker purples) in DAs that are more proximate to schools. Hence, all
DAs have higher values within Hamilton Central and those more proximate
to schools in less-urban communities. This trend persists across 2011,
2016 and 2016, aligning with intuition: populations with shorter travel
times to with relatively greater schools are calculated to have high
school seat spatial availability. Motorized populations have shorter
travel times at DAs further in distance from schools, while
non-motorized populations are only competitive at proximate distances.

The bottom two rows of \(v_i\) plots in Figure~\ref{fig-Fig7} (yellows
to greens) demonstrate the spatial availability per mode-using student
population e.g., \(V_i^m\) divided by the number of students aged 5-14
using a motorized mode or non-motorized mode. It is notable that for
motorized populations, their spatial availability per student is above 1
(green) within the center of the city. Conversely, this rate is only
available to non-motorized populations in DAs that are in less-densely
populated rural areas that are school proximate and very few pockets of
Hamilton Central in 2011 and 2016. In 2021, even fewer DAs within
Hamilton Central have non-motorized spatial availability rates above 1.

\begin{figure}[H]

\centering{

\includegraphics[width=1.2\textwidth,height=\textheight]{Manuscript_files/figure-pdf/fig-Fig7-1.pdf}

}

\caption{\label{fig-Fig7}Multimodal spatial availability (purples) and
spatial availability per mode-using student (diverging reds and greens)
aggregated at the DA level for 2011, 2016 and 2021. Scales are
represented in deciles}

\end{figure}%

To further explore these changes, the difference in \(V_i\) (first two
rows) and difference in \(v_i\) (second two rows) between 2011-2016,
2016-2021 and overall between 2011-2021 are displayed in
Figure~\ref{fig-Fig8} along with schools that changed during that time
period. 2011-2016 and 2016-2021 add together to equal 2011-2021 changes
quite clearly, e.g.~in the third row, the 2011-16 plots a small decrease
within the core of the city, and 2016-2021 demonstrates a more
significant decrease in the core and western peripheral communities.
Together, 2011-2021 plot reflects the changes in these two plots.

A few interesting spatial trends can be observed in
Figure~\ref{fig-Fig8}. First, DAs in proximity to schools that closed
(square points) see losses in \(V_i\) (reds) for both motorized and
especially non-motorized populations. For non-motorized populations, all
closed school resulted in a decrease in \(V_i\). The majority of school
closures took place in the central and eastern parts of Hamilton's urban
area (Hamilton Central), as well as in rural areas of western Hamilton
(Flamborough). Hence, losses characterise those neighbourhoods.
Secondly, gains in \(V_i\) (blues) are seen where schools are opened
(diamond points) or expanded (triangle points), especially for motorized
populations. These gains are most notably in the more recently urbanized
southeastern area of Hamilton (Glanbrook), along with a few pockets of
DAs in proximity to new or expanded schools in Hamilton Central.
Thirdly, the concentration of students in proximity to schools and the
extent of their OTGC matters from the perspecitve of modal advantage. In
areas with fewer proximate school options and lower OTGC relative to the
amount of students, a change in a school results in a big impact on
spatial availability values. For instance, in the community of Ancaster,
one school expanded in the rural area in this community between
2011-2016. The motorized student population, especially in rural DAs,
saw gains in spatial availability while non-motorized populations saw
losses. Students in proximity to the schools in Ancaster have high
non-motorized spatial availability, but when the school was expanded
(and due to an under-served population) motorized populations saw an
increase in spatial availability at the direct expense of non-motorized
populations' spatial availability.

Exploring changes in \(V_i\) provides a holistic view of spatial
availability for a given DA. However, if the focus is on the ratio of
spatial availability per student, the bottom two rows in
Figure~\ref{fig-Fig8} offer more insight. For instance, a community may
have had an initially average rate of school seats per student but is
expecting an increase in student population so it expands a school.
However, due to an increase in population, the spatial availability per
student overall still decreases between these two time periods. This
case appears to occur in Ancaster between 2011 to 2016: while motorized
\(V_i\) increased, motorized \(v_i\) slightly decreased, and
non-motorized \(v_i\) decreased relatively more than non-motorized
\(V_i\). As shown in Figure~\ref{fig-Fig3}, the student population and
the number of students per residential unit increased more rapidly than
the expanded school could accommodate, leading to a decrease in \(v_i\)
values in DAs proximate to expanded schools.

Similarly, in the southeastern community of Glanbrook, student
populations grew from 2011 to 2021, and new schools opened. However,
\(v_i\) rates decreased because OTGC did not keep pace with student
population growth. As another example, in Hamilton Central and the
eastern community of Stoney Creek, changes in \(V_i\) are unpatterned,
while \(v_i\) demonstrate more uniform decreases, with the pattern
varying by modal-range for motorized versus non-motorized populations.
In these communities, the student population remained relatively stable,
though OTGC decreased overall. \(v_i\) is illuminating to consider when
comparing changes in spatial availability rates.

\begin{figure}[H]

\centering{

\includegraphics[width=1.2\textwidth,height=\textheight]{Manuscript_files/figure-pdf/fig-Fig8-1.pdf}

}

\caption{\label{fig-Fig8}Change in multimodal spatial availability and
spatial availability per mode-using student from 2011 to 2016, 2016 to
2021, and together from 2011 to 2021. Change is the subtraction of
parcel-level results in that given year and summed for each DA in the
2021 Canadian Census DA grid. Scales are represented in deciles.}

\end{figure}%

Changes in multimodal spatial availability is important: A summary of
spatial availability per Hamilton community and associated dimensions in
2021, along with the percentage change between 2011 and 2021, is
presented in Table~\ref{tbl-Tab1}.

\global\setlength{\Oldarrayrulewidth}{\arrayrulewidth}

\global\setlength{\Oldtabcolsep}{\tabcolsep}

\setlength{\tabcolsep}{2pt}

\renewcommand*{\arraystretch}{1.5}



\providecommand{\ascline}[3]{\noalign{\global\arrayrulewidth #1}\arrayrulecolor[HTML]{#2}\cline{#3}}

\begin{longtable}[c]{|p{0.75in}|p{0.75in}|p{0.75in}|p{0.75in}|p{0.75in}|p{0.75in}|p{0.75in}|p{0.75in}|p{0.75in}|p{0.75in}|p{0.75in}|p{0.75in}|p{0.75in}}

\caption{\label{tbl-Tab1}Spatial availability and associated dimensions
aggregated by communities in 2011 and how much they changed by 2021}

\tabularnewline

\hhline{>{\arrayrulecolor[HTML]{BEBEBE}\global\arrayrulewidth=1.5pt}->{\arrayrulecolor[HTML]{BEBEBE}\global\arrayrulewidth=1.5pt}->{\arrayrulecolor[HTML]{BEBEBE}\global\arrayrulewidth=1.5pt}->{\arrayrulecolor[HTML]{BEBEBE}\global\arrayrulewidth=1.5pt}->{\arrayrulecolor[HTML]{BEBEBE}\global\arrayrulewidth=1.5pt}->{\arrayrulecolor[HTML]{BEBEBE}\global\arrayrulewidth=1.5pt}->{\arrayrulecolor[HTML]{BEBEBE}\global\arrayrulewidth=1.5pt}->{\arrayrulecolor[HTML]{BEBEBE}\global\arrayrulewidth=1.5pt}->{\arrayrulecolor[HTML]{BEBEBE}\global\arrayrulewidth=1.5pt}->{\arrayrulecolor[HTML]{BEBEBE}\global\arrayrulewidth=1.5pt}->{\arrayrulecolor[HTML]{BEBEBE}\global\arrayrulewidth=1.5pt}->{\arrayrulecolor[HTML]{BEBEBE}\global\arrayrulewidth=1.5pt}->{\arrayrulecolor[HTML]{BEBEBE}\global\arrayrulewidth=1.5pt}-}

\multicolumn{1}{>{\centering}m{\dimexpr 0.75in+0\tabcolsep}}{\textcolor[HTML]{000000}{\fontsize{10}{10}\selectfont{\global\setmainfont{DejaVu Sans}{\textbf{}}}}} & \multicolumn{1}{>{\centering}m{\dimexpr 0.75in+0\tabcolsep}}{\textcolor[HTML]{000000}{\fontsize{10}{10}\selectfont{\global\setmainfont{DejaVu Sans}{\textbf{Hamilton\ C.}}}}} & \multicolumn{1}{>{\centering}m{\dimexpr 0.75in+0\tabcolsep}}{\textcolor[HTML]{000000}{\fontsize{10}{10}\selectfont{\global\setmainfont{DejaVu Sans}{\textbf{\%\ Δ}}}}} & \multicolumn{1}{>{\centering}m{\dimexpr 0.75in+0\tabcolsep}}{\textcolor[HTML]{000000}{\fontsize{10}{10}\selectfont{\global\setmainfont{DejaVu Sans}{\textbf{Dundas}}}}} & \multicolumn{1}{>{\centering}m{\dimexpr 0.75in+0\tabcolsep}}{\textcolor[HTML]{000000}{\fontsize{10}{10}\selectfont{\global\setmainfont{DejaVu Sans}{\textbf{\%\ Δ}}}}} & \multicolumn{1}{>{\centering}m{\dimexpr 0.75in+0\tabcolsep}}{\textcolor[HTML]{000000}{\fontsize{10}{10}\selectfont{\global\setmainfont{DejaVu Sans}{\textbf{Stoney\ Creek}}}}} & \multicolumn{1}{>{\centering}m{\dimexpr 0.75in+0\tabcolsep}}{\textcolor[HTML]{000000}{\fontsize{10}{10}\selectfont{\global\setmainfont{DejaVu Sans}{\textbf{\%\ Δ}}}}} & \multicolumn{1}{>{\centering}m{\dimexpr 0.75in+0\tabcolsep}}{\textcolor[HTML]{000000}{\fontsize{10}{10}\selectfont{\global\setmainfont{DejaVu Sans}{\textbf{Glanbrook}}}}} & \multicolumn{1}{>{\centering}m{\dimexpr 0.75in+0\tabcolsep}}{\textcolor[HTML]{000000}{\fontsize{10}{10}\selectfont{\global\setmainfont{DejaVu Sans}{\textbf{\%\ Δ}}}}} & \multicolumn{1}{>{\centering}m{\dimexpr 0.75in+0\tabcolsep}}{\textcolor[HTML]{000000}{\fontsize{10}{10}\selectfont{\global\setmainfont{DejaVu Sans}{\textbf{Flamborough}}}}} & \multicolumn{1}{>{\centering}m{\dimexpr 0.75in+0\tabcolsep}}{\textcolor[HTML]{000000}{\fontsize{10}{10}\selectfont{\global\setmainfont{DejaVu Sans}{\textbf{\%\ Δ}}}}} & \multicolumn{1}{>{\centering}m{\dimexpr 0.75in+0\tabcolsep}}{\textcolor[HTML]{000000}{\fontsize{10}{10}\selectfont{\global\setmainfont{DejaVu Sans}{\textbf{Ancaster}}}}} & \multicolumn{1}{>{\centering}m{\dimexpr 0.75in+0\tabcolsep}}{\textcolor[HTML]{000000}{\fontsize{10}{10}\selectfont{\global\setmainfont{DejaVu Sans}{\textbf{\%\ Δ}}}}} \\

\noalign{\global\arrayrulewidth 0pt}\arrayrulecolor[HTML]{000000}

\hhline{>{\arrayrulecolor[HTML]{BEBEBE}\global\arrayrulewidth=1.5pt}->{\arrayrulecolor[HTML]{BEBEBE}\global\arrayrulewidth=1.5pt}->{\arrayrulecolor[HTML]{BEBEBE}\global\arrayrulewidth=1.5pt}->{\arrayrulecolor[HTML]{BEBEBE}\global\arrayrulewidth=1.5pt}->{\arrayrulecolor[HTML]{BEBEBE}\global\arrayrulewidth=1.5pt}->{\arrayrulecolor[HTML]{BEBEBE}\global\arrayrulewidth=1.5pt}->{\arrayrulecolor[HTML]{BEBEBE}\global\arrayrulewidth=1.5pt}->{\arrayrulecolor[HTML]{BEBEBE}\global\arrayrulewidth=1.5pt}->{\arrayrulecolor[HTML]{BEBEBE}\global\arrayrulewidth=1.5pt}->{\arrayrulecolor[HTML]{BEBEBE}\global\arrayrulewidth=1.5pt}->{\arrayrulecolor[HTML]{BEBEBE}\global\arrayrulewidth=1.5pt}->{\arrayrulecolor[HTML]{BEBEBE}\global\arrayrulewidth=1.5pt}->{\arrayrulecolor[HTML]{BEBEBE}\global\arrayrulewidth=1.5pt}-}\endfirsthead 

\hhline{>{\arrayrulecolor[HTML]{BEBEBE}\global\arrayrulewidth=1.5pt}->{\arrayrulecolor[HTML]{BEBEBE}\global\arrayrulewidth=1.5pt}->{\arrayrulecolor[HTML]{BEBEBE}\global\arrayrulewidth=1.5pt}->{\arrayrulecolor[HTML]{BEBEBE}\global\arrayrulewidth=1.5pt}->{\arrayrulecolor[HTML]{BEBEBE}\global\arrayrulewidth=1.5pt}->{\arrayrulecolor[HTML]{BEBEBE}\global\arrayrulewidth=1.5pt}->{\arrayrulecolor[HTML]{BEBEBE}\global\arrayrulewidth=1.5pt}->{\arrayrulecolor[HTML]{BEBEBE}\global\arrayrulewidth=1.5pt}->{\arrayrulecolor[HTML]{BEBEBE}\global\arrayrulewidth=1.5pt}->{\arrayrulecolor[HTML]{BEBEBE}\global\arrayrulewidth=1.5pt}->{\arrayrulecolor[HTML]{BEBEBE}\global\arrayrulewidth=1.5pt}->{\arrayrulecolor[HTML]{BEBEBE}\global\arrayrulewidth=1.5pt}->{\arrayrulecolor[HTML]{BEBEBE}\global\arrayrulewidth=1.5pt}-}

\multicolumn{1}{>{\centering}m{\dimexpr 0.75in+0\tabcolsep}}{\textcolor[HTML]{000000}{\fontsize{10}{10}\selectfont{\global\setmainfont{DejaVu Sans}{\textbf{}}}}} & \multicolumn{1}{>{\centering}m{\dimexpr 0.75in+0\tabcolsep}}{\textcolor[HTML]{000000}{\fontsize{10}{10}\selectfont{\global\setmainfont{DejaVu Sans}{\textbf{Hamilton\ C.}}}}} & \multicolumn{1}{>{\centering}m{\dimexpr 0.75in+0\tabcolsep}}{\textcolor[HTML]{000000}{\fontsize{10}{10}\selectfont{\global\setmainfont{DejaVu Sans}{\textbf{\%\ Δ}}}}} & \multicolumn{1}{>{\centering}m{\dimexpr 0.75in+0\tabcolsep}}{\textcolor[HTML]{000000}{\fontsize{10}{10}\selectfont{\global\setmainfont{DejaVu Sans}{\textbf{Dundas}}}}} & \multicolumn{1}{>{\centering}m{\dimexpr 0.75in+0\tabcolsep}}{\textcolor[HTML]{000000}{\fontsize{10}{10}\selectfont{\global\setmainfont{DejaVu Sans}{\textbf{\%\ Δ}}}}} & \multicolumn{1}{>{\centering}m{\dimexpr 0.75in+0\tabcolsep}}{\textcolor[HTML]{000000}{\fontsize{10}{10}\selectfont{\global\setmainfont{DejaVu Sans}{\textbf{Stoney\ Creek}}}}} & \multicolumn{1}{>{\centering}m{\dimexpr 0.75in+0\tabcolsep}}{\textcolor[HTML]{000000}{\fontsize{10}{10}\selectfont{\global\setmainfont{DejaVu Sans}{\textbf{\%\ Δ}}}}} & \multicolumn{1}{>{\centering}m{\dimexpr 0.75in+0\tabcolsep}}{\textcolor[HTML]{000000}{\fontsize{10}{10}\selectfont{\global\setmainfont{DejaVu Sans}{\textbf{Glanbrook}}}}} & \multicolumn{1}{>{\centering}m{\dimexpr 0.75in+0\tabcolsep}}{\textcolor[HTML]{000000}{\fontsize{10}{10}\selectfont{\global\setmainfont{DejaVu Sans}{\textbf{\%\ Δ}}}}} & \multicolumn{1}{>{\centering}m{\dimexpr 0.75in+0\tabcolsep}}{\textcolor[HTML]{000000}{\fontsize{10}{10}\selectfont{\global\setmainfont{DejaVu Sans}{\textbf{Flamborough}}}}} & \multicolumn{1}{>{\centering}m{\dimexpr 0.75in+0\tabcolsep}}{\textcolor[HTML]{000000}{\fontsize{10}{10}\selectfont{\global\setmainfont{DejaVu Sans}{\textbf{\%\ Δ}}}}} & \multicolumn{1}{>{\centering}m{\dimexpr 0.75in+0\tabcolsep}}{\textcolor[HTML]{000000}{\fontsize{10}{10}\selectfont{\global\setmainfont{DejaVu Sans}{\textbf{Ancaster}}}}} & \multicolumn{1}{>{\centering}m{\dimexpr 0.75in+0\tabcolsep}}{\textcolor[HTML]{000000}{\fontsize{10}{10}\selectfont{\global\setmainfont{DejaVu Sans}{\textbf{\%\ Δ}}}}} \\

\noalign{\global\arrayrulewidth 0pt}\arrayrulecolor[HTML]{000000}

\hhline{>{\arrayrulecolor[HTML]{BEBEBE}\global\arrayrulewidth=1.5pt}->{\arrayrulecolor[HTML]{BEBEBE}\global\arrayrulewidth=1.5pt}->{\arrayrulecolor[HTML]{BEBEBE}\global\arrayrulewidth=1.5pt}->{\arrayrulecolor[HTML]{BEBEBE}\global\arrayrulewidth=1.5pt}->{\arrayrulecolor[HTML]{BEBEBE}\global\arrayrulewidth=1.5pt}->{\arrayrulecolor[HTML]{BEBEBE}\global\arrayrulewidth=1.5pt}->{\arrayrulecolor[HTML]{BEBEBE}\global\arrayrulewidth=1.5pt}->{\arrayrulecolor[HTML]{BEBEBE}\global\arrayrulewidth=1.5pt}->{\arrayrulecolor[HTML]{BEBEBE}\global\arrayrulewidth=1.5pt}->{\arrayrulecolor[HTML]{BEBEBE}\global\arrayrulewidth=1.5pt}->{\arrayrulecolor[HTML]{BEBEBE}\global\arrayrulewidth=1.5pt}->{\arrayrulecolor[HTML]{BEBEBE}\global\arrayrulewidth=1.5pt}->{\arrayrulecolor[HTML]{BEBEBE}\global\arrayrulewidth=1.5pt}-}\endhead



\multicolumn{1}{>{\centering}m{\dimexpr 0.75in+0\tabcolsep}}{\textcolor[HTML]{000000}{\fontsize{10}{10}\selectfont{\global\setmainfont{DejaVu Sans}{LIM-AT}}}} & \multicolumn{1}{>{\centering}m{\dimexpr 0.75in+0\tabcolsep}}{\textcolor[HTML]{000000}{\fontsize{10}{10}\selectfont{\global\setmainfont{DejaVu Sans}{23.8516970}}}} & \multicolumn{1}{>{\cellcolor[HTML]{F4FBF3}\centering}m{\dimexpr 0.75in+0\tabcolsep}}{\textcolor[HTML]{000000}{\fontsize{10}{10}\selectfont{\global\setmainfont{DejaVu Sans}{13.2\%}}}} & \multicolumn{1}{>{\centering}m{\dimexpr 0.75in+0\tabcolsep}}{\textcolor[HTML]{000000}{\fontsize{10}{10}\selectfont{\global\setmainfont{DejaVu Sans}{11.2879110}}}} & \multicolumn{1}{>{\cellcolor[HTML]{F6FCF5}\centering}m{\dimexpr 0.75in+0\tabcolsep}}{\textcolor[HTML]{000000}{\fontsize{10}{10}\selectfont{\global\setmainfont{DejaVu Sans}{10.6\%}}}} & \multicolumn{1}{>{\centering}m{\dimexpr 0.75in+0\tabcolsep}}{\textcolor[HTML]{000000}{\fontsize{10}{10}\selectfont{\global\setmainfont{DejaVu Sans}{10.8019420}}}} & \multicolumn{1}{>{\cellcolor[HTML]{F7FCF7}\centering}m{\dimexpr 0.75in+0\tabcolsep}}{\textcolor[HTML]{000000}{\fontsize{10}{10}\selectfont{\global\setmainfont{DejaVu Sans}{9.1\%}}}} & \multicolumn{1}{>{\centering}m{\dimexpr 0.75in+0\tabcolsep}}{\textcolor[HTML]{000000}{\fontsize{10}{10}\selectfont{\global\setmainfont{DejaVu Sans}{8.3083330}}}} & \multicolumn{1}{>{\cellcolor[HTML]{FFFAF8}\centering}m{\dimexpr 0.75in+0\tabcolsep}}{\textcolor[HTML]{000000}{\fontsize{10}{10}\selectfont{\global\setmainfont{DejaVu Sans}{-2.5\%}}}} & \multicolumn{1}{>{\centering}m{\dimexpr 0.75in+0\tabcolsep}}{\textcolor[HTML]{000000}{\fontsize{10}{10}\selectfont{\global\setmainfont{DejaVu Sans}{7.2350900}}}} & \multicolumn{1}{>{\cellcolor[HTML]{FAFDFA}\centering}m{\dimexpr 0.75in+0\tabcolsep}}{\textcolor[HTML]{000000}{\fontsize{10}{10}\selectfont{\global\setmainfont{DejaVu Sans}{5.4\%}}}} & \multicolumn{1}{>{\centering}m{\dimexpr 0.75in+0\tabcolsep}}{\textcolor[HTML]{000000}{\fontsize{10}{10}\selectfont{\global\setmainfont{DejaVu Sans}{7.0989650}}}} & \multicolumn{1}{>{\cellcolor[HTML]{EAF7E9}\centering}m{\dimexpr 0.75in+0\tabcolsep}}{\textcolor[HTML]{000000}{\fontsize{10}{10}\selectfont{\global\setmainfont{DejaVu Sans}{24.9\%}}}} \\

\noalign{\global\arrayrulewidth 0pt}\arrayrulecolor[HTML]{000000}





\multicolumn{1}{>{\centering}m{\dimexpr 0.75in+0\tabcolsep}}{\textcolor[HTML]{000000}{\fontsize{10}{10}\selectfont{\global\setmainfont{DejaVu Sans}{Kid\ pop.}}}} & \multicolumn{1}{>{\centering}m{\dimexpr 0.75in+0\tabcolsep}}{\textcolor[HTML]{000000}{\fontsize{10}{10}\selectfont{\global\setmainfont{DejaVu Sans}{35310.5960000}}}} & \multicolumn{1}{>{\cellcolor[HTML]{FFF9F7}\centering}m{\dimexpr 0.75in+0\tabcolsep}}{\textcolor[HTML]{000000}{\fontsize{10}{10}\selectfont{\global\setmainfont{DejaVu Sans}{-3.2\%}}}} & \multicolumn{1}{>{\centering}m{\dimexpr 0.75in+0\tabcolsep}}{\textcolor[HTML]{000000}{\fontsize{10}{10}\selectfont{\global\setmainfont{DejaVu Sans}{2590.0000000}}}} & \multicolumn{1}{>{\cellcolor[HTML]{FFE3D9}\centering}m{\dimexpr 0.75in+0\tabcolsep}}{\textcolor[HTML]{000000}{\fontsize{10}{10}\selectfont{\global\setmainfont{DejaVu Sans}{-14.7\%}}}} & \multicolumn{1}{>{\centering}m{\dimexpr 0.75in+0\tabcolsep}}{\textcolor[HTML]{000000}{\fontsize{10}{10}\selectfont{\global\setmainfont{DejaVu Sans}{7514.7510000}}}} & \multicolumn{1}{>{\cellcolor[HTML]{EEF9ED}\centering}m{\dimexpr 0.75in+0\tabcolsep}}{\textcolor[HTML]{000000}{\fontsize{10}{10}\selectfont{\global\setmainfont{DejaVu Sans}{20.0\%}}}} & \multicolumn{1}{>{\centering}m{\dimexpr 0.75in+0\tabcolsep}}{\textcolor[HTML]{000000}{\fontsize{10}{10}\selectfont{\global\setmainfont{DejaVu Sans}{2639.6530000}}}} & \multicolumn{1}{>{\cellcolor[HTML]{A2DB9F}\centering}m{\dimexpr 0.75in+0\tabcolsep}}{\textcolor[HTML]{000000}{\fontsize{10}{10}\selectfont{\global\setmainfont{DejaVu Sans}{108.9\%}}}} & \multicolumn{1}{>{\centering}m{\dimexpr 0.75in+0\tabcolsep}}{\textcolor[HTML]{000000}{\fontsize{10}{10}\selectfont{\global\setmainfont{DejaVu Sans}{5365.0000000}}}} & \multicolumn{1}{>{\cellcolor[HTML]{F8FCF7}\centering}m{\dimexpr 0.75in+0\tabcolsep}}{\textcolor[HTML]{000000}{\fontsize{10}{10}\selectfont{\global\setmainfont{DejaVu Sans}{8.7\%}}}} & \multicolumn{1}{>{\centering}m{\dimexpr 0.75in+0\tabcolsep}}{\textcolor[HTML]{000000}{\fontsize{10}{10}\selectfont{\global\setmainfont{DejaVu Sans}{4845.0000000}}}} & \multicolumn{1}{>{\cellcolor[HTML]{F5FBF4}\centering}m{\dimexpr 0.75in+0\tabcolsep}}{\textcolor[HTML]{000000}{\fontsize{10}{10}\selectfont{\global\setmainfont{DejaVu Sans}{12.4\%}}}} \\

\noalign{\global\arrayrulewidth 0pt}\arrayrulecolor[HTML]{000000}





\multicolumn{1}{>{\centering}m{\dimexpr 0.75in+0\tabcolsep}}{\textcolor[HTML]{000000}{\fontsize{10}{10}\selectfont{\global\setmainfont{DejaVu Sans}{OTGC}}}} & \multicolumn{1}{>{\centering}m{\dimexpr 0.75in+0\tabcolsep}}{\textcolor[HTML]{000000}{\fontsize{10}{10}\selectfont{\global\setmainfont{DejaVu Sans}{39889.0820000}}}} & \multicolumn{1}{>{\cellcolor[HTML]{FFECE6}\centering}m{\dimexpr 0.75in+0\tabcolsep}}{\textcolor[HTML]{000000}{\fontsize{10}{10}\selectfont{\global\setmainfont{DejaVu Sans}{-9.8\%}}}} & \multicolumn{1}{>{\centering}m{\dimexpr 0.75in+0\tabcolsep}}{\textcolor[HTML]{000000}{\fontsize{10}{10}\selectfont{\global\setmainfont{DejaVu Sans}{2377.0380000}}}} & \multicolumn{1}{>{\cellcolor[HTML]{FFFFFF}\centering}m{\dimexpr 0.75in+0\tabcolsep}}{\textcolor[HTML]{000000}{\fontsize{10}{10}\selectfont{\global\setmainfont{DejaVu Sans}{0.0\%}}}} & \multicolumn{1}{>{\centering}m{\dimexpr 0.75in+0\tabcolsep}}{\textcolor[HTML]{000000}{\fontsize{10}{10}\selectfont{\global\setmainfont{DejaVu Sans}{9010.1960000}}}} & \multicolumn{1}{>{\cellcolor[HTML]{FFF1EB}\centering}m{\dimexpr 0.75in+0\tabcolsep}}{\textcolor[HTML]{000000}{\fontsize{10}{10}\selectfont{\global\setmainfont{DejaVu Sans}{-7.6\%}}}} & \multicolumn{1}{>{\centering}m{\dimexpr 0.75in+0\tabcolsep}}{\textcolor[HTML]{000000}{\fontsize{10}{10}\selectfont{\global\setmainfont{DejaVu Sans}{1313.5220000}}}} & \multicolumn{1}{>{\cellcolor[HTML]{91D58F}\centering}m{\dimexpr 0.75in+0\tabcolsep}}{\textcolor[HTML]{000000}{\fontsize{10}{10}\selectfont{\global\setmainfont{DejaVu Sans}{127.6\%}}}} & \multicolumn{1}{>{\centering}m{\dimexpr 0.75in+0\tabcolsep}}{\textcolor[HTML]{000000}{\fontsize{10}{10}\selectfont{\global\setmainfont{DejaVu Sans}{4590.7010000}}}} & \multicolumn{1}{>{\cellcolor[HTML]{FFF1EB}\centering}m{\dimexpr 0.75in+0\tabcolsep}}{\textcolor[HTML]{000000}{\fontsize{10}{10}\selectfont{\global\setmainfont{DejaVu Sans}{-7.5\%}}}} & \multicolumn{1}{>{\centering}m{\dimexpr 0.75in+0\tabcolsep}}{\textcolor[HTML]{000000}{\fontsize{10}{10}\selectfont{\global\setmainfont{DejaVu Sans}{3386.8900000}}}} & \multicolumn{1}{>{\cellcolor[HTML]{F6FBF5}\centering}m{\dimexpr 0.75in+0\tabcolsep}}{\textcolor[HTML]{000000}{\fontsize{10}{10}\selectfont{\global\setmainfont{DejaVu Sans}{11.1\%}}}} \\

\noalign{\global\arrayrulewidth 0pt}\arrayrulecolor[HTML]{000000}

\hhline{>{\arrayrulecolor[HTML]{BEBEBE}\global\arrayrulewidth=1pt}->{\arrayrulecolor[HTML]{BEBEBE}\global\arrayrulewidth=1pt}->{\arrayrulecolor[HTML]{BEBEBE}\global\arrayrulewidth=1pt}->{\arrayrulecolor[HTML]{BEBEBE}\global\arrayrulewidth=1pt}->{\arrayrulecolor[HTML]{BEBEBE}\global\arrayrulewidth=1pt}->{\arrayrulecolor[HTML]{BEBEBE}\global\arrayrulewidth=1pt}->{\arrayrulecolor[HTML]{BEBEBE}\global\arrayrulewidth=1pt}->{\arrayrulecolor[HTML]{BEBEBE}\global\arrayrulewidth=1pt}->{\arrayrulecolor[HTML]{BEBEBE}\global\arrayrulewidth=1pt}->{\arrayrulecolor[HTML]{BEBEBE}\global\arrayrulewidth=1pt}->{\arrayrulecolor[HTML]{BEBEBE}\global\arrayrulewidth=1pt}->{\arrayrulecolor[HTML]{BEBEBE}\global\arrayrulewidth=1pt}->{\arrayrulecolor[HTML]{BEBEBE}\global\arrayrulewidth=1pt}-}



\multicolumn{13}{>{\centering}m{\dimexpr 9.75in+24\tabcolsep}}{\textcolor[HTML]{000000}{\fontsize{10}{10}\selectfont{\global\setmainfont{DejaVu Sans}{\textbf{Motorized}}}}} \\

\noalign{\global\arrayrulewidth 0pt}\arrayrulecolor[HTML]{000000}

\hhline{>{\arrayrulecolor[HTML]{BEBEBE}\global\arrayrulewidth=1pt}->{\arrayrulecolor[HTML]{BEBEBE}\global\arrayrulewidth=1pt}->{\arrayrulecolor[HTML]{BEBEBE}\global\arrayrulewidth=1pt}->{\arrayrulecolor[HTML]{BEBEBE}\global\arrayrulewidth=1pt}->{\arrayrulecolor[HTML]{BEBEBE}\global\arrayrulewidth=1pt}->{\arrayrulecolor[HTML]{BEBEBE}\global\arrayrulewidth=1pt}->{\arrayrulecolor[HTML]{BEBEBE}\global\arrayrulewidth=1pt}->{\arrayrulecolor[HTML]{BEBEBE}\global\arrayrulewidth=1pt}->{\arrayrulecolor[HTML]{BEBEBE}\global\arrayrulewidth=1pt}->{\arrayrulecolor[HTML]{BEBEBE}\global\arrayrulewidth=1pt}->{\arrayrulecolor[HTML]{BEBEBE}\global\arrayrulewidth=1pt}->{\arrayrulecolor[HTML]{BEBEBE}\global\arrayrulewidth=1pt}->{\arrayrulecolor[HTML]{BEBEBE}\global\arrayrulewidth=1pt}-}



\multicolumn{1}{>{\centering}m{\dimexpr 0.75in+0\tabcolsep}}{\textcolor[HTML]{000000}{\fontsize{10}{10}\selectfont{\global\setmainfont{DejaVu Sans}{V\ (mt)}}}} & \multicolumn{1}{>{\centering}m{\dimexpr 0.75in+0\tabcolsep}}{\textcolor[HTML]{000000}{\fontsize{10}{10}\selectfont{\global\setmainfont{DejaVu Sans}{42527.7500000}}}} & \multicolumn{1}{>{\cellcolor[HTML]{FFE9E1}\centering}m{\dimexpr 0.75in+0\tabcolsep}}{\textcolor[HTML]{000000}{\fontsize{10}{10}\selectfont{\global\setmainfont{DejaVu Sans}{-11.6\%}}}} & \multicolumn{1}{>{\centering}m{\dimexpr 0.75in+0\tabcolsep}}{\textcolor[HTML]{000000}{\fontsize{10}{10}\selectfont{\global\setmainfont{DejaVu Sans}{1952.2590000}}}} & \multicolumn{1}{>{\cellcolor[HTML]{FFD4C5}\centering}m{\dimexpr 0.75in+0\tabcolsep}}{\textcolor[HTML]{000000}{\fontsize{10}{10}\selectfont{\global\setmainfont{DejaVu Sans}{-22.5\%}}}} & \multicolumn{1}{>{\centering}m{\dimexpr 0.75in+0\tabcolsep}}{\textcolor[HTML]{000000}{\fontsize{10}{10}\selectfont{\global\setmainfont{DejaVu Sans}{6328.0100000}}}} & \multicolumn{1}{>{\cellcolor[HTML]{F7FCF6}\centering}m{\dimexpr 0.75in+0\tabcolsep}}{\textcolor[HTML]{000000}{\fontsize{10}{10}\selectfont{\global\setmainfont{DejaVu Sans}{9.9\%}}}} & \multicolumn{1}{>{\centering}m{\dimexpr 0.75in+0\tabcolsep}}{\textcolor[HTML]{000000}{\fontsize{10}{10}\selectfont{\global\setmainfont{DejaVu Sans}{1531.3080000}}}} & \multicolumn{1}{>{\cellcolor[HTML]{9DD99A}\centering}m{\dimexpr 0.75in+0\tabcolsep}}{\textcolor[HTML]{000000}{\fontsize{10}{10}\selectfont{\global\setmainfont{DejaVu Sans}{114.8\%}}}} & \multicolumn{1}{>{\centering}m{\dimexpr 0.75in+0\tabcolsep}}{\textcolor[HTML]{000000}{\fontsize{10}{10}\selectfont{\global\setmainfont{DejaVu Sans}{3382.5270000}}}} & \multicolumn{1}{>{\cellcolor[HTML]{FFFBF9}\centering}m{\dimexpr 0.75in+0\tabcolsep}}{\textcolor[HTML]{000000}{\fontsize{10}{10}\selectfont{\global\setmainfont{DejaVu Sans}{-2.3\%}}}} & \multicolumn{1}{>{\centering}m{\dimexpr 0.75in+0\tabcolsep}}{\textcolor[HTML]{000000}{\fontsize{10}{10}\selectfont{\global\setmainfont{DejaVu Sans}{4086.6160000}}}} & \multicolumn{1}{>{\cellcolor[HTML]{F9FDF9}\centering}m{\dimexpr 0.75in+0\tabcolsep}}{\textcolor[HTML]{000000}{\fontsize{10}{10}\selectfont{\global\setmainfont{DejaVu Sans}{7.2\%}}}} \\

\noalign{\global\arrayrulewidth 0pt}\arrayrulecolor[HTML]{000000}





\multicolumn{1}{>{\centering}m{\dimexpr 0.75in+0\tabcolsep}}{\textcolor[HTML]{000000}{\fontsize{10}{10}\selectfont{\global\setmainfont{DejaVu Sans}{v\ (mt)}}}} & \multicolumn{1}{>{\centering}m{\dimexpr 0.75in+0\tabcolsep}}{\textcolor[HTML]{000000}{\fontsize{10}{10}\selectfont{\global\setmainfont{DejaVu Sans}{1.2393060}}}} & \multicolumn{1}{>{\cellcolor[HTML]{FFEEE8}\centering}m{\dimexpr 0.75in+0\tabcolsep}}{\textcolor[HTML]{000000}{\fontsize{10}{10}\selectfont{\global\setmainfont{DejaVu Sans}{-8.9\%}}}} & \multicolumn{1}{>{\centering}m{\dimexpr 0.75in+0\tabcolsep}}{\textcolor[HTML]{000000}{\fontsize{10}{10}\selectfont{\global\setmainfont{DejaVu Sans}{0.7590198}}}} & \multicolumn{1}{>{\cellcolor[HTML]{FFEEE7}\centering}m{\dimexpr 0.75in+0\tabcolsep}}{\textcolor[HTML]{000000}{\fontsize{10}{10}\selectfont{\global\setmainfont{DejaVu Sans}{-9.0\%}}}} & \multicolumn{1}{>{\centering}m{\dimexpr 0.75in+0\tabcolsep}}{\textcolor[HTML]{000000}{\fontsize{10}{10}\selectfont{\global\setmainfont{DejaVu Sans}{0.8369173}}}} & \multicolumn{1}{>{\cellcolor[HTML]{FFF1EC}\centering}m{\dimexpr 0.75in+0\tabcolsep}}{\textcolor[HTML]{000000}{\fontsize{10}{10}\selectfont{\global\setmainfont{DejaVu Sans}{-7.1\%}}}} & \multicolumn{1}{>{\centering}m{\dimexpr 0.75in+0\tabcolsep}}{\textcolor[HTML]{000000}{\fontsize{10}{10}\selectfont{\global\setmainfont{DejaVu Sans}{0.5823797}}}} & \multicolumn{1}{>{\cellcolor[HTML]{FCFEFC}\centering}m{\dimexpr 0.75in+0\tabcolsep}}{\textcolor[HTML]{000000}{\fontsize{10}{10}\selectfont{\global\setmainfont{DejaVu Sans}{3.1\%}}}} & \multicolumn{1}{>{\centering}m{\dimexpr 0.75in+0\tabcolsep}}{\textcolor[HTML]{000000}{\fontsize{10}{10}\selectfont{\global\setmainfont{DejaVu Sans}{0.6312921}}}} & \multicolumn{1}{>{\cellcolor[HTML]{FFECE5}\centering}m{\dimexpr 0.75in+0\tabcolsep}}{\textcolor[HTML]{000000}{\fontsize{10}{10}\selectfont{\global\setmainfont{DejaVu Sans}{-10.1\%}}}} & \multicolumn{1}{>{\centering}m{\dimexpr 0.75in+0\tabcolsep}}{\textcolor[HTML]{000000}{\fontsize{10}{10}\selectfont{\global\setmainfont{DejaVu Sans}{0.8491653}}}} & \multicolumn{1}{>{\cellcolor[HTML]{FFF6F2}\centering}m{\dimexpr 0.75in+0\tabcolsep}}{\textcolor[HTML]{000000}{\fontsize{10}{10}\selectfont{\global\setmainfont{DejaVu Sans}{-5.0\%}}}} \\

\noalign{\global\arrayrulewidth 0pt}\arrayrulecolor[HTML]{000000}

\hhline{>{\arrayrulecolor[HTML]{BEBEBE}\global\arrayrulewidth=1pt}->{\arrayrulecolor[HTML]{BEBEBE}\global\arrayrulewidth=1pt}->{\arrayrulecolor[HTML]{BEBEBE}\global\arrayrulewidth=1pt}->{\arrayrulecolor[HTML]{BEBEBE}\global\arrayrulewidth=1pt}->{\arrayrulecolor[HTML]{BEBEBE}\global\arrayrulewidth=1pt}->{\arrayrulecolor[HTML]{BEBEBE}\global\arrayrulewidth=1pt}->{\arrayrulecolor[HTML]{BEBEBE}\global\arrayrulewidth=1pt}->{\arrayrulecolor[HTML]{BEBEBE}\global\arrayrulewidth=1pt}->{\arrayrulecolor[HTML]{BEBEBE}\global\arrayrulewidth=1pt}->{\arrayrulecolor[HTML]{BEBEBE}\global\arrayrulewidth=1pt}->{\arrayrulecolor[HTML]{BEBEBE}\global\arrayrulewidth=1pt}->{\arrayrulecolor[HTML]{BEBEBE}\global\arrayrulewidth=1pt}->{\arrayrulecolor[HTML]{BEBEBE}\global\arrayrulewidth=1pt}-}



\multicolumn{13}{>{\centering}m{\dimexpr 9.75in+24\tabcolsep}}{\textcolor[HTML]{000000}{\fontsize{10}{10}\selectfont{\global\setmainfont{DejaVu Sans}{\textbf{Non-motorized}}}}} \\

\noalign{\global\arrayrulewidth 0pt}\arrayrulecolor[HTML]{000000}

\hhline{>{\arrayrulecolor[HTML]{BEBEBE}\global\arrayrulewidth=1pt}->{\arrayrulecolor[HTML]{BEBEBE}\global\arrayrulewidth=1pt}->{\arrayrulecolor[HTML]{BEBEBE}\global\arrayrulewidth=1pt}->{\arrayrulecolor[HTML]{BEBEBE}\global\arrayrulewidth=1pt}->{\arrayrulecolor[HTML]{BEBEBE}\global\arrayrulewidth=1pt}->{\arrayrulecolor[HTML]{BEBEBE}\global\arrayrulewidth=1pt}->{\arrayrulecolor[HTML]{BEBEBE}\global\arrayrulewidth=1pt}->{\arrayrulecolor[HTML]{BEBEBE}\global\arrayrulewidth=1pt}->{\arrayrulecolor[HTML]{BEBEBE}\global\arrayrulewidth=1pt}->{\arrayrulecolor[HTML]{BEBEBE}\global\arrayrulewidth=1pt}->{\arrayrulecolor[HTML]{BEBEBE}\global\arrayrulewidth=1pt}->{\arrayrulecolor[HTML]{BEBEBE}\global\arrayrulewidth=1pt}->{\arrayrulecolor[HTML]{BEBEBE}\global\arrayrulewidth=1pt}-}



\multicolumn{1}{>{\centering}m{\dimexpr 0.75in+0\tabcolsep}}{\textcolor[HTML]{000000}{\fontsize{10}{10}\selectfont{\global\setmainfont{DejaVu Sans}{V\ (nmt)}}}} & \multicolumn{1}{>{\centering}m{\dimexpr 0.75in+0\tabcolsep}}{\textcolor[HTML]{000000}{\fontsize{10}{10}\selectfont{\global\setmainfont{DejaVu Sans}{631.8817250}}}} & \multicolumn{1}{>{\cellcolor[HTML]{FFEAE2}\centering}m{\dimexpr 0.75in+0\tabcolsep}}{\textcolor[HTML]{000000}{\fontsize{10}{10}\selectfont{\global\setmainfont{DejaVu Sans}{-11.0\%}}}} & \multicolumn{1}{>{\centering}m{\dimexpr 0.75in+0\tabcolsep}}{\textcolor[HTML]{000000}{\fontsize{10}{10}\selectfont{\global\setmainfont{DejaVu Sans}{28.7632500}}}} & \multicolumn{1}{>{\cellcolor[HTML]{FFF1EC}\centering}m{\dimexpr 0.75in+0\tabcolsep}}{\textcolor[HTML]{000000}{\fontsize{10}{10}\selectfont{\global\setmainfont{DejaVu Sans}{-7.1\%}}}} & \multicolumn{1}{>{\centering}m{\dimexpr 0.75in+0\tabcolsep}}{\textcolor[HTML]{000000}{\fontsize{10}{10}\selectfont{\global\setmainfont{DejaVu Sans}{65.1592800}}}} & \multicolumn{1}{>{\cellcolor[HTML]{FFB39B}\centering}m{\dimexpr 0.75in+0\tabcolsep}}{\textcolor[HTML]{000000}{\fontsize{10}{10}\selectfont{\global\setmainfont{DejaVu Sans}{-39.5\%}}}} & \multicolumn{1}{>{\centering}m{\dimexpr 0.75in+0\tabcolsep}}{\textcolor[HTML]{000000}{\fontsize{10}{10}\selectfont{\global\setmainfont{DejaVu Sans}{2.3855420}}}} & \multicolumn{1}{>{\cellcolor[HTML]{7CCD7C}\centering}m{\dimexpr 0.75in+0\tabcolsep}}{\textcolor[HTML]{000000}{\fontsize{10}{10}\selectfont{\global\setmainfont{DejaVu Sans}{1274.9\%}}}} & \multicolumn{1}{>{\centering}m{\dimexpr 0.75in+0\tabcolsep}}{\textcolor[HTML]{000000}{\fontsize{10}{10}\selectfont{\global\setmainfont{DejaVu Sans}{12.2516440}}}} & \multicolumn{1}{>{\cellcolor[HTML]{FFFDFC}\centering}m{\dimexpr 0.75in+0\tabcolsep}}{\textcolor[HTML]{000000}{\fontsize{10}{10}\selectfont{\global\setmainfont{DejaVu Sans}{-1.0\%}}}} & \multicolumn{1}{>{\centering}m{\dimexpr 0.75in+0\tabcolsep}}{\textcolor[HTML]{000000}{\fontsize{10}{10}\selectfont{\global\setmainfont{DejaVu Sans}{18.5170610}}}} & \multicolumn{1}{>{\cellcolor[HTML]{FF7957}\centering}m{\dimexpr 0.75in+0\tabcolsep}}{\textcolor[HTML]{000000}{\fontsize{10}{10}\selectfont{\global\setmainfont{DejaVu Sans}{-67.6\%}}}} \\

\noalign{\global\arrayrulewidth 0pt}\arrayrulecolor[HTML]{000000}





\multicolumn{1}{>{\centering}m{\dimexpr 0.75in+0\tabcolsep}}{\textcolor[HTML]{000000}{\fontsize{10}{10}\selectfont{\global\setmainfont{DejaVu Sans}{v\ (nmt)}}}} & \multicolumn{1}{>{\centering}m{\dimexpr 0.75in+0\tabcolsep}}{\textcolor[HTML]{000000}{\fontsize{10}{10}\selectfont{\global\setmainfont{DejaVu Sans}{0.7054674}}}} & \multicolumn{1}{>{\cellcolor[HTML]{FFE6DD}\centering}m{\dimexpr 0.75in+0\tabcolsep}}{\textcolor[HTML]{000000}{\fontsize{10}{10}\selectfont{\global\setmainfont{DejaVu Sans}{-13.0\%}}}} & \multicolumn{1}{>{\centering}m{\dimexpr 0.75in+0\tabcolsep}}{\textcolor[HTML]{000000}{\fontsize{10}{10}\selectfont{\global\setmainfont{DejaVu Sans}{1.5790158}}}} & \multicolumn{1}{>{\cellcolor[HTML]{FFDFD4}\centering}m{\dimexpr 0.75in+0\tabcolsep}}{\textcolor[HTML]{000000}{\fontsize{10}{10}\selectfont{\global\setmainfont{DejaVu Sans}{-16.6\%}}}} & \multicolumn{1}{>{\centering}m{\dimexpr 0.75in+0\tabcolsep}}{\textcolor[HTML]{000000}{\fontsize{10}{10}\selectfont{\global\setmainfont{DejaVu Sans}{0.8967578}}}} & \multicolumn{1}{>{\cellcolor[HTML]{FFCFBE}\centering}m{\dimexpr 0.75in+0\tabcolsep}}{\textcolor[HTML]{000000}{\fontsize{10}{10}\selectfont{\global\setmainfont{DejaVu Sans}{-25.3\%}}}} & \multicolumn{1}{>{\centering}m{\dimexpr 0.75in+0\tabcolsep}}{\textcolor[HTML]{000000}{\fontsize{10}{10}\selectfont{\global\setmainfont{DejaVu Sans}{1.6565846}}}} & \multicolumn{1}{>{\cellcolor[HTML]{FFD0BF}\centering}m{\dimexpr 0.75in+0\tabcolsep}}{\textcolor[HTML]{000000}{\fontsize{10}{10}\selectfont{\global\setmainfont{DejaVu Sans}{-24.8\%}}}} & \multicolumn{1}{>{\centering}m{\dimexpr 0.75in+0\tabcolsep}}{\textcolor[HTML]{000000}{\fontsize{10}{10}\selectfont{\global\setmainfont{DejaVu Sans}{1.8555444}}}} & \multicolumn{1}{>{\cellcolor[HTML]{FFE8DF}\centering}m{\dimexpr 0.75in+0\tabcolsep}}{\textcolor[HTML]{000000}{\fontsize{10}{10}\selectfont{\global\setmainfont{DejaVu Sans}{-12.2\%}}}} & \multicolumn{1}{>{\centering}m{\dimexpr 0.75in+0\tabcolsep}}{\textcolor[HTML]{000000}{\fontsize{10}{10}\selectfont{\global\setmainfont{DejaVu Sans}{0.8641681}}}} & \multicolumn{1}{>{\cellcolor[HTML]{FFDFD3}\centering}m{\dimexpr 0.75in+0\tabcolsep}}{\textcolor[HTML]{000000}{\fontsize{10}{10}\selectfont{\global\setmainfont{DejaVu Sans}{-17.0\%}}}} \\

\noalign{\global\arrayrulewidth 0pt}\arrayrulecolor[HTML]{000000}

\hhline{>{\arrayrulecolor[HTML]{BEBEBE}\global\arrayrulewidth=1.5pt}->{\arrayrulecolor[HTML]{BEBEBE}\global\arrayrulewidth=1.5pt}->{\arrayrulecolor[HTML]{BEBEBE}\global\arrayrulewidth=1.5pt}->{\arrayrulecolor[HTML]{BEBEBE}\global\arrayrulewidth=1.5pt}->{\arrayrulecolor[HTML]{BEBEBE}\global\arrayrulewidth=1.5pt}->{\arrayrulecolor[HTML]{BEBEBE}\global\arrayrulewidth=1.5pt}->{\arrayrulecolor[HTML]{BEBEBE}\global\arrayrulewidth=1.5pt}->{\arrayrulecolor[HTML]{BEBEBE}\global\arrayrulewidth=1.5pt}->{\arrayrulecolor[HTML]{BEBEBE}\global\arrayrulewidth=1.5pt}->{\arrayrulecolor[HTML]{BEBEBE}\global\arrayrulewidth=1.5pt}->{\arrayrulecolor[HTML]{BEBEBE}\global\arrayrulewidth=1.5pt}->{\arrayrulecolor[HTML]{BEBEBE}\global\arrayrulewidth=1.5pt}->{\arrayrulecolor[HTML]{BEBEBE}\global\arrayrulewidth=1.5pt}-}


\end{longtable}

\arrayrulecolor[HTML]{000000}

\global\setlength{\arrayrulewidth}{\Oldarrayrulewidth}

\global\setlength{\tabcolsep}{\Oldtabcolsep}

\renewcommand*{\arraystretch}{1}

Due to spatial availability's proportional allocation mechanism and this
cases' calculation at the parcel level, results can also be summarized
by community (Table~\ref{tbl-Tab1}). The majority of schools were closed
in Hamilton Central as seen in the largest reduction in OTGC by
percentage and magnitude. Hamilton Central represents 55\% of the 5-14
year old population in 2021. Though the population decreased a modest
amount from 2011 to 2021, the rate of school-seat spatial availability
decreased disproportionately more (population by -3.2\% and spatial
availability by -11.6\% for both modes). Schools and additional OTGC was
not sufficiently expanded to provide the same levels of spatial
availability as in 2011. This disproportionate decrease in spatial
availability to students is greater in Hamilton Central by magnitude
than any other community.

From the perspective of equity, Hamilton Central has the highest LIM-AT
prevalence, with 27\% of households LIM-AT in 2016, more than 3 times
greater than Ancaster, Flamborough and Glanbrook, communities with the
lowest level of LIM-AT prevalence. Is the disproportionate decrease in
spatial availability within Hamilton Central fair? In some ways it is:
communities with the lowest LIM-AT prevalence appear to benefit from
gains in motorized spatial availability. Glanbrook, Flamborough and
Ancaster together account for 27\% of the 5-14 year old population in
2021. However, they access 19\% of the spatial availability of school
seats. Hamilton Central currently captures 66\% of the spatial
availability for it's 55.0\% of the population. With student aged
population growing in other communities, OTGC should be expanded -
potentially to levels that Hamilton had in 2011 or beyond (1.2
school-seats per motorized mode using student and 0.7 school-seats per
active mode using student). However, should those gains come at a loss
to other communities, especially those with a significantly higher
proportion of households who are lower-income, a strong determinant of
transport poverty?

Furthermore, while there are gains in motorized spatial availability in
certain communities by some metrics, there are losses in non-motorized
spatial availability per student from 2011 to 2021 in all communities.
And the losses are drastic. Communities with high community averag
spatial availability per students with values above 1 school-seat per
student saw losses, as well as communities below this value.

\section{Discussion and conclusion}\label{discussion-and-conclusion}

In this paper, we compared how multimodal spatial availability changed
after a wave of school closures and consolidations in Hamilton. To do
so, we constructed the student-aged population, OTGC of schools and
their locations, and their travel behaviour to generate spatial
availability landscapes for the three study years. We aggregated the
resulting values at the DA and community level and descriptively
compared differences. We demonstrated that city-wide there are decreases
in \(V_i\) for both mode-using populations: the majority of students in
the city have access to fewer school-seats than they would have in 2011.
Furthermore, as the majority of closed schools are within the core of
the city: the more urban Hamilton Central and urban-areas of other
communities saw the largest amount of this decrease. And at a
local-scale, students in residences proximate to schools that closed
also saw dramatic reduce in non-motorized spatial availability values.

Evidently, the number of OTGC in the city decreased between 2011 and
2021, so some sort of decrease was to be expected. However, should that
decrease be felt hardest within neighbourhoods with the highest LIM-AT?
By students who have the potential to travel actively to school?
Normatively, our cities should be increasing spatial availability for
non-motorized mode users given the benefits of active travel and the
climate crisis. Hamilton's policies from 2011 through 2021 drastically
reduced non-motorized spatial availability, and gains in motorized
spatial availability especially in more rural areas. The proportion of
non-motorized mode use is low in both 2011 and 2021, however, the
closure of schools that are more proximate to student-population density
eliminates the potential of those trips ever becoming active.

Overall, the closure of 10\% of elementary schools in Hamilton between
2011 and 2021 resulted in 5\% decline in school-seat availability
city-wide ( 5\% fewer school-seats), but specifically a 90\% decline in
non-motorized spatial availability. Though non-motorized modal share was
low in both 2011 and 2016, the \emph{potential} for active transport
trips was significantly eroded as schools within a 27 minute walk
catchment were closed in the more densely populated and central areas of
the city. Further, the communities with the highest LIM-AT prevalence
were the most impacted: communities like Hamilton Central and Dundas saw
the most drastic loss in non-motorized spatial availability while
communities with lower LIM-AT prevalence saw gains in motorized spatial
availability (as a handful of rural schools opened). The motivation of
operational savings associated with public service consolidation for the
school boards did not account for the mobility-related burdens
\emph{certain} families are now saddled with. The decision to
consolidate schools, likely has increased the length of motorized
travel, though this would need to be further investigated. A decrease in
non-motorized trips will have daily impacts on students and their
families (Mandic et al. 2022; Pabayo et al. 2012), the broader community
(Pietrabissa 2023; Merrall, Higgins, and Páez 2024; Bittencourt and
Giannotti 2023) and the environment (Pantelaki, Claudia Caspani, and
Maggi 2024; Rong et al. 2022).

Like many other localities (J. Lee and Lubienski 2017; Autti and
Hyry-Beihammer 2014; Beuchert et al. 2018), Hamilton is not immune to
top-down operational efficiency assessments that determine if community
infrastructure is underutilized and closed. From the perspective of the
provincial government of Ontario, the school closures which occurred in
Hamilton resulted in operational savings-per-student. We demonstrate how
these decisions resulted in both families and students paying a price
through multimodal spatial availability.

By quantifying the spatial availability, environmental, and
active-travel implications of school closure/consolidation policies in
Hamilton, our paper offers a methodology for spatial policy analysis
scenario. The presented methodology can be used by researchers and
decision-makers to plan and evaluate equitable and sustainable urban
planning policies from a spatial perspective. At the core of the method
is spatial availability (Soukhov et al. 2023; Soukhov et al. 2024), a
singly-constrained accessibility measure that can be calculated at the
finest spatial resolution (in our paper, parcel-level) and aggregated at
whichever spatial unit is most meaningful for interpretation. We urge
policy-makers to view the `spatial availability' of public resources
from a per-capita perspective, plan capacity and location of service
provision with the cost of travel (and associated implications such as
GHG emissions, active transportation mode, safety of active travel
modes, etc.) that sufficiently serves the target population. In this
work, we suggest and assume a benchmark of 1.0 school-seats per student
which accounts for sufficiency. Spatial availability can be used to
obtain this benchmark since the total number of opportunities (i.e.,
school capacity) is preserved in the region of analysis unlike
unconstrained forms of accessibility measurement e.g., Hansen-type
measure (Hansen 1959). As such, the assigned spatial availability can be
divided by population at each zone to yield an interpretable per capita
measurement.

Ultimately, this case study furnishes evidence that consolidation of
schools was particularly ill-timed: with the advent of the COVID-19
pandemic, the education system lacked the resiliency to accommodate
reduced classroom capacities and left parents who once lived within
active transportation distance to schools relying on motorized
transportation for their children. These top-down austerity measures
left the Hamilton local school system more vulnerable to the impacts of
COVID-19 by undervaluing the societal role of schools in neighbourhoods.
By identifying the impact of school consolidation policies in Ontario we
anticipate that researchers and decision-makers will have better
information to center the wellbeing of all residents, including
students, and the planet in urban planning policies.

As with all studies, our work is associated with limitations on how
results should be interpreted. The calculated spatial availability
assumes any residential parcel can have a student (at the DA rate of
students per residential unit) so all parcels are assigned a DA rate
weighted by how many residential units are contained within the parcel.
Each parcel can also visit any school, though travel times based on
observed origin-destination behaviour (from the TTS) are more likely
through the use of the empirical travel impedance functions. In this
way, the spatial availability demonstrates what \emph{potential
interaction} a student could have based on historic and average travel
behaviour and shortest network-path travel times if residing in point in
space. This representation is average, and by design obfuscates
considerations that could make non-average and shortest-path travel
unrealistic such as safety and built-environment concerns (C. Lee et al.
2021; Yumita et al. 2021) nor considers significant non-transportation
factors that impact fulsome school accessibility like quality
(Bittencourt and Giannotti 2023). Further, this study does not consider
transit, school-bus, or cycling modes which have implications on how the
GHG emissions estimation can be interpreted. Transit and school bus
modes often taken more circuitous routes (Yumita et al. 2021) but pool
students thereby reducing emissions. Cycling reduces travel time over
the same walking distance at no extra emitted GHG. Lastly, this study
also ascribes a static emission factor to represent motorized travel
minutes based on an average diesel passenger car. In reality, GHG
emissions emitted per minute of driving varies extensively depending on
operating conditions and vehicle characteristics (Soukhov and Mohamed
2022). In these ways, the assumptions made are to illustrate
\emph{potential} and \emph{average} changes that resulted in a backdrop
of school consolidation/closures to illustrate accessibility-related
impacts for the city of Hamilton.

\section{Declarations}\label{declarations}

\textbf{Ethical approval} Not applicable

\textbf{Participate consent} Not applicable

\textbf{Competing interests} The authors declare there are no personal
or financial conflicts of interest.

\textbf{Authors' contributions} A.S., A.P., C.D.H., and M.M. all
contributed to the study conception and design. Material preparation,
data collection and formal analysis were performed by A.S., A.P, and
C.D.H. All visuals and the first draft of the manuscript was written by
A.S. A.S., A.P., C.D.H., and M.M. commented on earlier versions of the
manuscript. A.S., A.P., C.D.H., and M.M. reviewed the final manuscript.

\textbf{Funding declaration} This work was supported by the Mobilizing
Justice Partnership and the Canada Graduate Scholarships---Doctoral
Program from the Social Sciences and Humanities Research Council of
Canada (SSHRC).

\textbf{Availability of data and materials} The manuscript text, code,
analysis and data supporting this work is available in the lead author's
GitHub repository :
https://github.com/soukhova/School-closures-accessibility-impacts

\pagebreak

\section{References}\label{references}

\phantomsection\label{refs}
\begin{CSLReferences}{1}{0}
\bibitem[\citeproctext]{ref-auditorgeneralofontarioAnnualReport20152015}
Auditor General of Ontario. 2015. {``Annual Report 2015.''} Office of
the Auditor General of Ontario.
\url{https://www.auditor.on.ca/en/content/annualreports/arreports/en15/2015AR_en_final.pdf}.

\bibitem[\citeproctext]{ref-AuttiHyryBeihammer2014}
Autti, Outi, and Eeva Kaisa Hyry-Beihammer. 2014. {``School Closures in
Rural Finnish Communities.''} \emph{Journal of Research in Rural
Education} 29 (1): 1--17.

\bibitem[\citeproctext]{ref-batista_estimation_2019}
Batista, S. F. A., Ludovic Leclercq, and Nikolas Geroliminis. 2019.
{``Estimation of Regional Trip Length Distributions for the Calibration
of the Aggregated Network Traffic Models.''} \emph{Transportation
Research Part B: Methodological} 122 (April): 192--217.
\url{https://doi.org/10.1016/j.trb.2019.02.009}.

\bibitem[\citeproctext]{ref-von_Bergmann_Shkolnik_Jacobs_2021}
Bergmann, J von, D Shkolnik, and A Jacobs. 2021. \emph{Cancensus: R
Package to Access, Retrieve, and Work with Canadian Census Data and
Geography}.

\bibitem[\citeproctext]{ref-Beuchert_Humlum_Nielsen_Smith_2018}
Beuchert, Louise, Maria Knoth Humlum, Helena Skyt Nielsen, and Nina
Smith. 2018. {``The Short-Term Effects of School Consolidation on
Student Achievement: Evidence of Disruption?''} \emph{Economics of
Education Review} 65 (August): 31--47.
\url{https://doi.org/10.1016/j.econedurev.2018.05.004}.

\bibitem[\citeproctext]{ref-bierbaumMobilityJusticeLinking2021}
Bierbaum, Ariel H., Alex Karner, and Jesus M. Barajas. 2021. {``Toward
{Mobility Justice}: {Linking Transportation} and {Education Equity} in
the {Context} of {School Choice}.''} \emph{Journal of the American
Planning Association} 87 (2): 197--210.
\url{https://doi.org/10.1080/01944363.2020.1803104}.

\bibitem[\citeproctext]{ref-bittencourtEvaluatingAccessibilityAvailability2023}
Bittencourt, Tainá A., and Mariana Giannotti. 2023. {``Evaluating the
Accessibility and Availability of Public Services to Reduce Inequalities
in Everyday Mobility.''} \emph{Transportation Research Part A: Policy
and Practice} 177 (November): 103833.
\url{https://doi.org/10.1016/j.tra.2023.103833}.

\bibitem[\citeproctext]{ref-brunsdonOpeningPractice2021}
Brunsdon, Chris, and Alexis Comber. 2021. {``Opening Practice:
Supporting Reproducibility and Critical Spatial Data Science.''}
\emph{Journal of Geographical Systems} 23 (4): 477--96.
\url{https://doi.org/10.1007/s10109-020-00334-2}.

\bibitem[\citeproctext]{ref-buckWhy2019}
BUCK, NAOMI. 2019. {``Why Did Our Children Stop Walking to School?''}
\emph{The Globe and Mail}, October, O1.
\url{http://global.factiva.com/redir/default.aspx?P=sa&an=GLOB000020191019efaj00011&cat=a&ep=ASE}.

\bibitem[\citeproctext]{ref-burgess2011parental}
Burgess, Simon, Ellen Greaves, Anna Vignoles, and Deborah Wilson. 2011.
{``Parental Choice of Primary School in England: What Types of School Do
Different Types of Family Really Have Available to Them?''} \emph{Policy
Studies} 32 (5): 531--47.

\bibitem[\citeproctext]{ref-butlerClosureRideau2019a}
Butler, Jesse K., Ruth G. Kane, and Fiona R. Cooligan. 2019. {``The
{Closure} of {Rideau High School}: {A Case Study} in the {Political
Economy} of {Urban Education} in {Ontario}.''} \emph{Canadian Journal of
Educational Administration and Policy}, no. 191.

\bibitem[\citeproctext]{ref-Statistics_Canada_2011}
Canada, Statistics. 2011. {``Census of Canda Census Profiles: Median
After-Tax Household Income, 5 to 9 Years, 10 to 14 Years, 15 to 19
Years, for Census Metroplian Areas and Dissimination Areas.''}

\bibitem[\citeproctext]{ref-Statistics_Canada_2016}
---------. 2016. {``Census of Canda Census Profiles: Median After-Tax
Household Income, 5 to 9 Years, 10 to 14 Years, 15 to 19 Years, for
Census Metroplian Areas and Dissimination Areas.''}

\bibitem[\citeproctext]{ref-Statistics_Canada_2021}
---------. 2021. {``Census of Canda Census Profiles: Median After-Tax
Household Income, 5 to 9 Years, 10 to 14 Years, 15 to 19 Years, for
Census Metroplian Areas and Dissimination Areas.''}

\bibitem[\citeproctext]{ref-Christiaanse_2020}
Christiaanse, Suzan. 2020. {``Rural Facility Decline: A Longitudinal
Accessibility Analysis Questioning the Focus of Dutch
Depopulation-Policy.''} \emph{Applied Geography} 121 (August): 102251.
\url{https://doi.org/10.1016/j.apgeog.2020.102251}.

\bibitem[\citeproctext]{ref-conveyalConveyalR5Routing2022}
Conveyal. (2015) 2022. {``Conveyal {R5 Routing Engine}.''} {Conveyal}.
\url{https://github.com/conveyal/r5}.

\bibitem[\citeproctext]{ref-craggsAxeCouldFall2012}
Craggs, Samantha. 2012. {``Axe Could Fall on More Hamilton Schools,
Projections Show.''} \emph{{CBC} News}, October.
\url{https://www.cbc.ca/news/canada/hamilton/headlines/axe-could-fall-on-more-hamilton-schools-projections-show-1.1183565}.

\bibitem[\citeproctext]{ref-craggsBoardEvaluatingFuture2013}
---------. 2013. {``Board Evaluating Future of 76 Public Elementary
Schools.''} \emph{{CBC} News}, February.
\url{https://www.cbc.ca/news/canada/hamilton/headlines/board-evaluating-future-of-76-public-elementary-schools-1.1347581}.

\bibitem[\citeproctext]{ref-Dai_Wang_Zhang_Liao_Liu_2019}
Dai, Te-qi, Liang Wang, Yu-chao Zhang, Cong Liao, and Zheng-bing Liu.
2019. {``The Cost of School Consolidation Policy: Implications from
Decomposing School Commuting Distances in Yanqing, Beijing.''}
\emph{Applied Spatial Analysis and Policy} 12 (2): 191--204.
\url{https://doi.org/10.1007/s12061-017-9238-2}.

\bibitem[\citeproctext]{ref-data_management_group_tts_2018}
Data Management Group. 2018. {``{TTS} - {Transportation} {Tomorrow}
{Survey} 2016.''}
\url{http://dmg.utoronto.ca/transportation-tomorrow-survey/tts-introduction}.

\bibitem[\citeproctext]{ref-fitdistrplus_2015}
Delignette-Muller, Marie Laure, and Christophe Dutang. 2015.
{``{fitdistrplus}: An {R} Package for Fitting Distributions.''}
\emph{Journal of Statistical Software} 64 (4): 1--34.
\url{https://www.jstatsoft.org/article/view/v064i04}.

\bibitem[\citeproctext]{ref-desjardinsFramingActive2024}
Desjardins, Elise, Jason Lam, Darcy Reynard, Damian Collins, E. Owen D.
Waygood, and Antonio Páez. 2024. {``Framing Active School Travel in
{Ontario}, or How Spinach Is Good for You.''} \emph{Transportation
Research Part A: Policy and Practice} 180 (February): 103953.
\url{https://doi.org/10.1016/j.tra.2024.103953}.

\bibitem[\citeproctext]{ref-elmurrWalkingAccessibilityParks2021}
El-Murr, Karl, Arianne Robillard, Owen Waygood, and Geneviève Boisjoly.
2021. {``Walking Accessibility to Parks: Considering Number of Parks,
Surface Area and Type of Activities.''} \emph{Findings}, October.
\url{https://doi.org/10.32866/001c.27479}.

\bibitem[\citeproctext]{ref-faoOntarioSchoolBoards2023}
FAO. 2023. {``Ontario School Boards: Enrolment, Finances and Student
Outcomes.''} 978-1-4868-7600-6. Financial Accountability Office of
Ontario.
\url{https://www.fao-on.org/en/Blog/Publications/FA2207schoolboards}.

\bibitem[\citeproctext]{ref-geofabrikOntarioOpenStreetMapGeofabrik2022}
Geofabrik. 2022. {``Ontario {OpenStreetMap} - {Geofabrik Download
Server}.''} 2022.
\url{https://download.geofabrik.de/north-america/canada/ontario.html}.

\bibitem[\citeproctext]{ref-governmentofcanadaDictionaryCensusPopulation2017}
Government of Canada, Statistics Canada. 2017. {``Dictionary, Census of
Population, 2016 - Low-Income Measure, After Tax ({LIM}-{AT}).''} May 3,
2017.
\url{https://www12.statcan.gc.ca/census-recensement/2016/ref/dict/fam021-eng.cfm}.

\bibitem[\citeproctext]{ref-governmentofcanadaProfileTableCensus2022}
---------. 2022a. {``Profile Table, Census Profile, 2021 Census of
Population - Hamilton, City (c) {[}Census Subdivision{]}, Ontario.''}
February 9, 2022.
\url{https://www12.statcan.gc.ca/census-recensement/2021/dp-pd/prof/index.cfm?Lang=E}.

\bibitem[\citeproctext]{ref-governmentofcanadaDailyPandemicBenefits2022}
---------. 2022b. {``The Daily --- Pandemic Benefits Cushion Losses for
Low Income Earners and Narrow Income Inequality -- After-Tax Income
Grows Across Canada Except in Alberta and Newfoundland and Labrador.''}
July 13, 2022.
\url{https://www150.statcan.gc.ca/n1/daily-quotidien/220713/dq220713d-eng.htm}.

\bibitem[\citeproctext]{ref-hamiltonEducationalInstitutions2024}
Hamilton. 2024. {``Educational Institutions.''} Open Hamilton.
\url{https://open.hamilton.ca/datasets/34794547fe3e46619b3ff92a574fd558_19/explore}.

\bibitem[\citeproctext]{ref-hammerWhy2012}
Hammer, Kate, and Tamara Baluja. 2012. {``Why Our Schools Need a Phys-Ed
Revolution; {An} Inactive Generation with Soaring Obesity Rates Shows
Schools Are Flunking When It Comes to Fitness.''} \emph{The Globe and
Mail}, May, A1.
\url{http://global.factiva.com/redir/default.aspx?P=sa&an=GLOB000020120524e85o0001r&cat=a&ep=ASE}.

\bibitem[\citeproctext]{ref-hansenHowAccessibilityShapes1959}
Hansen, Walter G. 1959. {``How {Accessibility Shapes Land Use}.''}
\emph{Journal of the American Institute of Planners} 25 (2): 73--76.
\url{https://doi.org/10.1080/01944365908978307}.

\bibitem[\citeproctext]{ref-hewkoMeasuringNeighbourhoodSpatial2002}
Hewko, Jared, Karen E Smoyer-Tomic, and M John Hodgson. 2002.
{``Measuring Neighbourhood Spatial Accessibility to Urban Amenities:
Does Aggregation Error Matter?''} \emph{Environment and Planning A:
Economy and Space} 34 (7): 1185--1206.
\url{https://doi.org/10.1068/a34171}.

\bibitem[\citeproctext]{ref-horbachov_theoretical_2018}
Horbachov, Peter, and Stanislav Svichynskyi. 2018. {``Theoretical
Substantiation of Trip Length Distribution for Home-Based Work Trips in
Urban Transit Systems.''} \emph{Journal of Transport and Land Use} 11
(1): 593--632. \url{https://www.jstor.org/stable/26622420}.

\bibitem[\citeproctext]{ref-hu_2019_measuring}
Hu, Yujie, and Joni Downs. 2019. {``Measuring and Visualizing
Place-Based Space-Time Job Accessibility.''} Journal Article.
\emph{Journal of Transport Geography} 74: 278--88.
\url{https://doi.org/10.1016/j.jtrangeo.2018.12.002}.

\bibitem[\citeproctext]{ref-hwdsbLongTermFacilities2013}
HWDSB. 2013. {``Long Term Facilities Master Plan - Accomodation Review
Schedule.''} Hamilton, Ontario: Hosted by {CBC}.
\url{https://www.documentcloud.org/documents/603032-hwdsb-schools-for-review}.

\bibitem[\citeproctext]{ref-HWDSB_2019}
HWDSB. 2019. {``Transportation: Policy No. 3.10.''}
\url{https://www.hamiltonschoolbus.ca/policies/hwdsb/HWDSB_Transportation_Policy.pdf}.

\bibitem[\citeproctext]{ref-hwdsb2023LongTermFacilities2023}
HWDSB. 2023. {``2023 Long-Term Facilities Master Plan - Section 1.7:
Accommodation Strategy Schedule.''} Hamilton, Ontario.
\url{https://www.hwdsb.on.ca/wp-content/uploads/2023/10/LTFMP-1.7-Accommodation-Strategy-Schedule-2023.pdf}.

\bibitem[\citeproctext]{ref-irwinSchoolClosure2012}
Irwin, Bill, and Mark Seasons. 2012. {``School {Closure Decision-Making
Processes Problems} and {Prospects}.''} \emph{Canadian Journal of Urban
Research} 21 (2): 45--67.

\bibitem[\citeproctext]{ref-josephMeasuringPotentialPhysical1982}
Joseph, Alun E., and Peter R. Bantock. 1982. {``Measuring Potential
Physical Accessibility to General Practitioners in Rural Areas: A Method
and Case Study.''} \emph{Social Science \& Medicine} 16 (1): 85--90.
\url{https://doi.org/10.1016/0277-9536(82)90428-2}.

\bibitem[\citeproctext]{ref-kaneParcelsPointsProximity2020}
Kane, Kevin, and Young-An Kim. 2020. {``Parcels, Points, and Proximity:
Can Exhaustive Sources of Big Data Improve Measurement in Cities?''}
\emph{Environment and Planning B: Urban Analytics and City Science} 47
(4): 695--715. \url{https://doi.org/10.1177/2399808318797135}.

\bibitem[\citeproctext]{ref-kelobonye2020measuring}
Kelobonye, Keone, Heng Zhou, Gary McCarney, and Jianhong Xia. 2020.
{``Measuring the Accessibility and Spatial Equity of Urban Services
Under Competition Using the Cumulative Opportunities Measure.''} Journal
Article. \emph{Journal of Transport Geography} 85: 102706.
https://doi.org/\url{https://doi.org/10.1016/j.jtrangeo.2020.102706}.

\bibitem[\citeproctext]{ref-kleinhuisSeriousConcernsSchool2013}
Kleinhuis, Amanda. 2013. {``Serious Concerns about School Accommodation
Review Process - Raise the Hammer.''} 2013.
\url{https://www.raisethehammer.org/article/1979/serious_concerns_about_school_accommodation_review_process}.

\bibitem[\citeproctext]{ref-kwanSpaceTimeIntegralMeasures1998}
Kwan, Mei-Po. 1998. {``Space-Time and Integral Measures of Individual
Accessibility: A Comparative Analysis Using a Point-Based Framework.''}
\emph{Geographical Analysis} 30 (3): 191--216.
\url{https://doi.org/10.1111/j.1538-4632.1998.tb00396.x}.

\bibitem[\citeproctext]{ref-leeNeighborhoodEnvironmentsUtilitarian2021}
Lee, C, C Lee, OT Stewart, HA Carlos, A Adachi-Mejia, EM Berke, and MP
Doescher. 2021. {``Neighborhood Environments and Utilitarian Walking
Among Older Vs. Younger Rural Adults.''} \emph{{FRONTIERS} {IN} {PUBLIC}
{HEALTH}} 9 (June). \url{https://doi.org/10.3389/fpubh.2021.634751}.

\bibitem[\citeproctext]{ref-Lee_Lubienski_2017}
Lee, Jin, and Christopher Lubienski. 2017. {``The Impact of School
Closures on Equity of Access in Chicago.''} \emph{Education and Urban
Society} 49 (1): 53--80. \url{https://doi.org/10.1177/0013124516630601}.

\bibitem[\citeproctext]{ref-luoMeasuresSpatialAccessibility2003}
Luo, Wei, and Fahui Wang. 2003. {``Measures of {Spatial Accessibility}
to {Health Care} in a {GIS Environment}: {Synthesis} and a {Case Study}
in the {Chicago Region}.''} \emph{Environment and Planning B: Planning
and Design} 30 (6): 865--84. \url{https://doi.org/10.1068/b29120}.

\bibitem[\citeproctext]{ref-mackenzieBlueprintFixOntario2018}
Mackenzie, Hugh. 2018. {``A Blueprint to Fix Ontario's Education Funding
Formula.''} \emph{Canadian Centre for Policy Alternatives {\textbar}
Ontario}.
\url{https://policyalternatives.ca/sites/default/files/uploads/publications/Ontario\%20Office/2018/03/Course\%20Correction.pdf}.

\bibitem[\citeproctext]{ref-mandicAdolescentsPerceptionsWalking2022}
Mandic, Sandra, Enrique García Bengoechea, Debbie Hopkins, Kirsten
Coppell, and John C. Spence. 2022. {``Adolescents' Perceptions of
Walking and Cycling to School Differ Based on How Far They Live from
School.''} \emph{Journal of Transport \& Health} 24 (March): 101316.
\url{https://doi.org/10.1016/j.jth.2021.101316}.

\bibitem[\citeproctext]{ref-marquesAccessibilityPrimarySchools2020}
Marques, João Lourenço, Jan Wolf, and Fillipe Feitosa. 2020.
{``Accessibility to Primary Schools in Portugal: A Case of Spatial
Inequity?''} \emph{Regional Science Policy \& Practice}, July,
rsp3.12303. \url{https://doi.org/10.1111/rsp3.12303}.

\bibitem[\citeproctext]{ref-merlin2017competition}
Merlin, Louis A., and Lingqian Hu. 2017. {``Does Competition Matter in
Measures of Job Accessibility? Explaining Employment in Los Angeles.''}
Journal Article. \emph{Journal of Transport Geography} 64: 77--88.
\url{https://doi.org/10.1016/j.jtrangeo.2017.08.009}.

\bibitem[\citeproctext]{ref-Merrall_2021}
Merrall, John. 2021. {``The Effect of School Closures on House Prices in
Hamilton, Ontario.''} PhD thesis, McMaster University; School of Earth,
Environment,; Society.

\bibitem[\citeproctext]{ref-MerrallWhatsASchool2024}
Merrall, John, Christopher D. Higgins, and Antonio Páez. 2024. {``What's
a School Worth to a Neighborhood? {A} Spatial Hedonic Analysis of
Property Prices in the Context of Accommodation Reviews in Ontario.''}
\emph{Geographical Analysis} 56 (2): 217--43.
\url{https://doi.org/10.1111/gean.12377}.

\bibitem[\citeproctext]{ref-ministryofeducationPupilAccommodationReview2006}
Ministry of Education. 2006. {``Pupil Accommodation Review Guidelines -
Memo.''} Toronto: Business; Finance Divisio.
\url{https://efis.fma.csc.gov.on.ca/faab/Memos/B2006/B_12.pdf}.

\bibitem[\citeproctext]{ref-moniruzzamanAccessibility2012}
Moniruzzaman, M., and A. Paez. 2012. {``Accessibility to Transit, by
Transit, and Mode Share: Application of a Logistic Model with Spatial
Filters.''} \emph{Journal of Transport Geography} 24 (September):
198--205. \url{https://doi.org/10.1016/j.jtrangeo.2012.02.006}.

\bibitem[\citeproctext]{ref-morencyHowMany2007}
Morency, C., M. Demers, and L. Lapierre. 2007. {``How Many Steps Do You
Have in Reserve? {Thoughts} and Measures about a Healthier Way to
Travel.''} \emph{Transportation Research Record} 2002: 1--6.
\url{https://doi.org/10.3141/2002-01}.

\bibitem[\citeproctext]{ref-morencyStepsReserve2009}
Morency, C., M. J. Roorda, and M. Demers. 2009. {``Steps in {Reserve
Comparing Latent Walk Trips} in {Toronto}, {Ontario}, and {Montreal},
{Quebec}, {Canada}.''} \emph{Transportation Research Record} 2140:
111--19. \url{https://doi.org/10.3141/2140-12}.

\bibitem[\citeproctext]{ref-morenomonroyPublicTransportSchool2018}
Moreno-Monroy, Ana I., Robin Lovelace, and Frederico R. Ramos. 2018.
{``Public Transport and School Location Impacts on Educational
Inequalities: Insights from São Paulo.''} \emph{Journal of Transport
Geography} 67 (February): 110--18.
\url{https://doi.org/10.1016/j.jtrangeo.2017.08.012}.

\bibitem[\citeproctext]{ref-mullerAssessmentSchoolClosures2011}
Müller, Sven. 2011. {``Assessment of School Closures in Urban Areas by
Simple Accessibility Measures.''} \emph{Erdkunde} 65 (4): 401--14.
\url{https://doi.org/10.3112/erdkunde.2011.04.06}.

\bibitem[\citeproctext]{ref-ontarioSchoolFacilityInventory2017}
Ontario. 2017. {``School Facility Inventory System ({SFIS}) - Dataset -
Ontario Data Catalogue.''} February 2017.
\url{https://data.ontario.ca/dataset/school-facility-inventory-system-sfis}.

\bibitem[\citeproctext]{ref-OpenStreetMap_2021}
OpenStreetMap. 2021.

\bibitem[\citeproctext]{ref-opsbaRightSchoolsRight2024}
OPSBA. 2024. {``The Right Schools in the Right Locations: {OPSBA} Asks
Government to Lift Moratorium on School Reviews.''} 2024.
\url{https://www.opsba.org/opsba_news/the-right-schools-in-the-right-locations-opsba-asks-government-to-lift-moratorium-on-school-reviews/}.

\bibitem[\citeproctext]{ref-pabayoImportanceActiveTransportation2012}
Pabayo, Roman, Katerina Maximova, John C. Spence, Kerry Vander Ploeg,
Biao Wu, and Paul J. Veugelers. 2012. {``The Importance of Active
Transportation to and from School for Daily Physical Activity Among
Children.''} \emph{Preventive Medicine} 55 (3): 196--200.
\url{https://doi.org/10.1016/j.ypmed.2012.06.008}.

\bibitem[\citeproctext]{ref-paezOpenSpatial2021c}
Páez, Antonio. 2021. {``Open Spatial Sciences: An Introduction.''}
\emph{Journal of Geographical Systems} 23 (4): 467--76.
\url{https://doi.org/10.1007/s10109-021-00364-4}.

\bibitem[\citeproctext]{ref-paez2019}
Páez, Antonio, Christopher D. Higgins, and Salvatore F. Vivona. 2019.
{``Demand and Level of Service Inflation in Floating Catchment Area
(FCA) Methods.''} Edited by Tayyab Ikram Shah. \emph{PLOS ONE} 14 (6):
e0218773. \url{https://doi.org/10.1371/journal.pone.0218773}.

\bibitem[\citeproctext]{ref-paez2012measuring}
Páez, A., D. M. Scott, and C. Morency. 2012. {``Measuring Accessibility:
Positive and Normative Implementations of Various Accessibility
Indicators.''} Journal Article. \emph{Journal of Transport Geography}
25: 141--53. \url{https://doi.org/10.1016/j.jtrangeo.2012.03.016}.

\bibitem[\citeproctext]{ref-pantelakiImpactHomeschoolCommuting2024}
Pantelaki, Evangelia, Anna Claudia Caspani, and Elena Maggi. 2024.
{``Impact of Home-School Commuting Mode Choice on Carbon Footprint and
Sustainable Transport Policy Scenarios.''} \emph{Case Studies on
Transport Policy} 15 (March): 101110.
\url{https://doi.org/10.1016/j.cstp.2023.101110}.

\bibitem[\citeproctext]{ref-pereiraGeographicAccessCOVID192021}
Pereira, Rafael H. M., Carlos Kauê Vieira Braga, Luciana Mendes Servo,
Bernardo Serra, Pedro Amaral, Nelson Gouveia, and Antonio Paez. 2021.
{``Geographic Access to {COVID}-19 Healthcare in Brazil Using a Balanced
Float Catchment Area Approach.''} \emph{Social Science \& Medicine} 273
(March): 113773. \url{https://doi.org/10.1016/j.socscimed.2021.113773}.

\bibitem[\citeproctext]{ref-Pereira2021r5r}
Pereira, Rafael H. M., Marcus Saraiva, Daniel Herszenhut, Carlos Kaue
Vieira Braga, and Matthew Wigginton Conway. 2021. {``R5r: Rapid
Realistic Routing on Multimodal Transport Networks with
r\textsuperscript{5} in r.''} \emph{Findings}, March.
\url{https://doi.org/10.32866/001c.21262}.

\bibitem[\citeproctext]{ref-pietrabissaSchoolAccessCity2023}
Pietrabissa, Giorgio. 2023. {``School Access and City Structure.''}
\url{https://giorgiopietrabissa.github.io/files/school_sorting.pdf}.

\bibitem[\citeproctext]{ref-pinchChangingGeographyPreschool1987}
Pinch, S. 1987. {``The Changing Geography of Preschool Services in
England Between 1977 and 1983.''} \emph{Environment and Planning C:
Government and Policy} 5 (4): 469--80.
\url{https://doi.org/10.1068/c050469}.

\bibitem[\citeproctext]{ref-pizzolQualifyingAccessibilityEducation2021}
Pizzol, Bruna, Mariana Giannotti, and Diego Bogado Tomasiello. 2021.
{``Qualifying Accessibility to Education to Investigate Spatial
Equity.''} \emph{Journal of Transport Geography} 96 (October): 103199.
\url{https://doi.org/10.1016/j.jtrangeo.2021.103199}.

\bibitem[\citeproctext]{ref-reyesalking2014}
Reyes, M., A. Paez, and C. Morency. 2014. {``Walking Accessibility to
Urban Parks by Children: {A} Case Study of {Montreal}.''}
\emph{Landscape and Urban Planning} 125 (May): 38--47.
\url{https://doi.org/10.1016/j.landurbplan.2014.02.002}.

\bibitem[\citeproctext]{ref-rongReviewResearchLowcarbon2022}
Rong, Peijun, Mei-Po Kwan, Yaochen Qin, and Zhicheng Zheng. 2022. {``A
Review of Research on Low-Carbon School Trips and Their Implications for
Human-Environment Relationship.''} \emph{Journal of Transport Geography}
99 (February): 103306.
\url{https://doi.org/10.1016/j.jtrangeo.2022.103306}.

\bibitem[\citeproctext]{ref-roseKids2013}
Rose, Lauren La. 2013. {``Kids Get {D} Minus on Physical Activity Report
Card as Fewer Walk or Bike to School.''} \emph{The Globe and Mail}, May,
L12.
\url{http://global.factiva.com/redir/default.aspx?P=sa&an=GLOB000020130522e95m00009&cat=a&ep=ASE}.

\bibitem[\citeproctext]{ref-Rosik_PulawskaObiedowska_Goliszek_2021}
Rosik, Piotr, Sabina Puławska-Obiedowska, and Sławomir Goliszek. 2021.
{``Public Transport Accessibility to Upper Secondary Schools Measured by
the Potential Quotient: The Case of Kraków.''} \emph{Moravian
Geographical Reports} 29 (1): 15--26.
\url{https://doi.org/10.2478/mgr-2021-0002}.

\bibitem[\citeproctext]{ref-sageman2022}
Sageman, Joseph. 2022. {``School Closures and Rural Population
Decline.''} \emph{Rural Sociology} 87 (3): 960--92.
\url{https://doi.org/10.1111/ruso.12437}.

\bibitem[\citeproctext]{ref-seasons2014ARCCatalogues2014}
Seasons, Mark. 2014a. {``2014 {ARC} Catalogues. 2014 English Public
{ARC} Inventory.''} 2014.
\url{https://uwaterloo.ca/school-closure-policy-research/research-findings/arc-data/2014-arc-catalogue}.

\bibitem[\citeproctext]{ref-seasonsSchoolClosurePolicy2014}
---------. 2014b. {``School Closure Policy {\textbar} School Closure
Policy Research.''} January 1, 2014.
\url{https://uwaterloo.ca/school-closure-policy-research/exploring-issue/school-closure-policy}.

\bibitem[\citeproctext]{ref-shenLocationCharacteristicsInnercity1998}
Shen, Q. 1998. {``Location Characteristics of Inner-City Neighborhoods
and Employment Accessibility of Low-Wage Workers.''} \emph{Environment
and Planning B: Planning and Design} 25 (3): 345--65.
\url{https://doi.org/10.1068/b250345}.

\bibitem[\citeproctext]{ref-soukhovOccupancyGHGEmissions2022}
Soukhov, Anastasia, and Moataz Mohamed. 2022. {``Occupancy and {GHG}
Emissions: Thresholds for Disruptive Transportation Modes and Emerging
Technologies.''} \emph{Transportation Research Part D: Transport and
Environment} 102 (January): 103127.
\url{https://doi.org/10.1016/j.trd.2021.103127}.

\bibitem[\citeproctext]{ref-soukhovTTS2016RDataSet2023}
Soukhov, Anastasia, and Antonio Páez. 2023. {``{TTS}2016R: A Data Set to
Study Population and Employment Patterns from the 2016 Transportation
Tomorrow Survey in the Greater Golden Horseshoe Area, Ontario,
Canada.''} \emph{Environment and Planning B: Urban Analytics and City
Science}, January, 23998083221146781.
\url{https://doi.org/10.1177/23998083221146781}.

\bibitem[\citeproctext]{ref-soukhovIntroducingSpatialAvailability2023}
Soukhov, Anastasia, Antonio Páez, Christopher D. Higgins, and Moataz
Mohamed. 2023. {``Introducing Spatial Availability, a Singly-Constrained
Measure of Competitive Accessibility \textbar{} {PLOS ONE}.''}
\emph{PLOS ONE}, 1--30.
https://doi.org/\href{https://\%20doi.org/10.1371/journal.pone.0278468}{https://
doi.org/10.1371/journal.pone.0278468}.

\bibitem[\citeproctext]{ref-soukhovMultimodalSpatialAvailability2024}
Soukhov, Anastasia, Javier Tarriño-Ortiz, Julio A. Soria-Lara, and
Antonio Páez. 2024. {``Multimodal Spatial Availability: A
Singly-Constrained Measure of Accessibility Considering Multiple
Modes.''} \emph{{PLOS} {ONE}} 19 (2): e0299077.
\url{https://doi.org/10.1371/journal.pone.0299077}.

\bibitem[\citeproctext]{ref-DMTI_Spatial_2015}
Spatial, DMTI. 2015. {``Building Footprints Region,''} September.
\url{http://geo.scholarsportal.info/\#r/details/_uri@=24}.

\bibitem[\citeproctext]{ref-Talen_2001}
Talen, Emily. 2001. {``School, Community, and Spatial Equity: An
Empirical Investigation of Access to Elementary Schools in West
Virginia.''} \emph{Annals of the Association of American Geographers} 91
(3): 465--86. \url{https://doi.org/10.1111/0004-5608.00254}.

\bibitem[\citeproctext]{ref-tdsbLiftingMinistryEducations2024}
TDSB. 2024. {``Lifting the Ministry of Education's Moratorium on School
Closures to Allow for Modernization Through Consolidation Monday April
15, 2024.''} April 2024.
\url{https://www.tdsb.on.ca/home/ctl/Details/mid/43824/itemid/280\#:~:text=At\%20a\%20recent\%20Board\%20meeting,under\%2Denrolled\%20schools\%20through\%20consolidation.}

\bibitem[\citeproctext]{ref-Teranet_2009}
Teranet. 2009. {``Hamilton Parcel/Land Use Data.''} Dept. of Planning;
Economic Development.

\bibitem[\citeproctext]{ref-thompsonItJustGalls2024}
Thompson, Carise, Patricia A. Collins, and Jennifer Dean. 2024. {``{`It
Just Galls Me, as a Taxpayer'}: Trust Implications of School Closure
Decision-Making in Ontario.''} \emph{Administration \& Society},
November, 00953997241296117.
\url{https://doi.org/10.1177/00953997241296117}.

\bibitem[\citeproctext]{ref-tiznadoaitkenPublicTransportAccessibility2021}
Tiznado-Aitken, I, JC Munoz, and R Hurtubia. 2021. {``Public Transport
Accessibility Accounting for Level of Service and Competition for Urban
Opportunities: An Equity Analysis for Education in Santiago de Chile.''}
\emph{{JOURNAL} {OF} {TRANSPORT} {GEOGRAPHY}} 90 (January).
\url{https://doi.org/10.1016/j.jtrangeo.2020.102919}.

\bibitem[\citeproctext]{ref-weibullAxiomaticApproachMeasurement1976}
Weibull, Jörgen W. 1976. {``An Axiomatic Approach to the Measurement of
Accessibility.''} \emph{Regional Science and Urban Economics} 6 (4):
357--79. \url{https://doi.org/10.1016/0166-0462(76)90031-4}.

\bibitem[\citeproctext]{ref-williamsDisparitiesAccessibilityPublic2014}
Williams, Shaun, and Fahui Wang. 2014. {``Disparities in Accessibility
of Public High Schools, in Metropolitan Baton Rouge, Louisiana
1990--2010.''} \emph{Urban Geography} 35 (7): 1066--83.
\url{https://doi.org/10.1080/02723638.2014.936668}.

\bibitem[\citeproctext]{ref-yumitaSchoolCommutingBarriers2021}
Yumita, FR, MZ Irawan, S Malkhamah, and MIH Kamal. 2021. {``School
Commuting: Barriers, Abilities and Strategies Toward Sustainable Public
Transport Systems in Yogyakarta, Indonesia.''} \emph{{SUSTAINABILITY}}
13 (16). \url{https://doi.org/10.3390/su13169372}.

\end{CSLReferences}




\end{document}
